%!TEX root = ../slides.tex
\begin{frame}[noframenumbering]{Research background}{Contributions}
\small
\begin{block}{\small \textbf{Contributions}}
\begin{itemize}
  \item An on-device mobility patterns detector that works with streams of raw data collected by smartphone's sensors (GPS and accelerometer).
  \item An on-device mobility analyzer that incrementally builds a model of user mobility from the detected mobility patterns.
  \item A cognitive controller inspired on CDSs that, based on the mobility information learned, dynamically adapts GPS sampling rate through power-aware policies. 
  \item A middleware with the previous modules embedded for easing the development of LBSs and MBSs for the Android mobile platform.
\end{itemize}
\end{block}
\end{frame}

\begin{frame}[noframenumbering]{State of the art}{Taxonomy of solutions}
\begin{figure}
  \centering
  \includegraphics[width=0.9\textwidth]{vectors/msa-stages-for-slides}
  \caption{Stages of mobile sensing apps.}
\end{figure}
\end{frame}

\begin{frame}[noframenumbering]{Theoretical framework}{Stay points}
\begin{block}{\small \textbf{Calculation}}
\begin{itemize}
  \item A GPS position fix $p$ is defined by latitude ($lat$), longitude ($lon$), and timestamp ($t$).

  \item A stay point ($\mathbf{sp}$) is calculated from a set of consecutive GPS fixes $\mathbf{P}=\{p_{m},p_{m+1},\ldots,p_{n}\}$, and $\delta_{distance}$ and $\delta_{time}$ thresholds. 
  
  \item Its composition must observe:
  \begin{columns}
  \begin{column}{0.5\textwidth}
  \begin{equation*}
    \left|p_{n}.t-p_{m}.t\right|\geq\delta_{time}
  \end{equation*}
  \end{column}

  \begin{column}{0.5\textwidth}
  \begin{equation*}
    \text{distance}(p_{m},p_{i})\leq\delta_{distance}, \forall m<i\leq n
  \end{equation*}
  \end{column}
  \end{columns}
  
  \item Its centroid coordinates are the arithmetical means:
  \begin{columns}
  \begin{column}{0.5\textwidth}
  \begin{equation*}
  \mathbf{sp}.lat = \frac{ \sum_{i=m}^{n}p_{i}.lat }{ |\mathbf{P}| }\label{eq:centroid-latitude}
  \end{equation*}
  \end{column}

  \begin{column}{0.5\textwidth}
  \begin{equation*}
  \mathbf{sp}.lon = \frac{ \sum_{i=m}^{n}p_{i}.lon }{ |\mathbf{P}| }\label{eq:centroid-longitude}
  \end{equation*}
  \end{column}
  \end{columns}
  \item The \emph{at} and \emph{dt} components are set to $p_m.t$ and $p_n.t$, respectively.
\end{itemize}
\end{block}
\end{frame}

{\aauwavesbg
  \begin{textblock*}{5cm}(0.3cm,0.3cm)
  \small
  \textbf{The architecture of the proposed system implemented in the Android software stack.}
  \end{textblock*}
\begin{frame}[plain,noframenumbering]
  \centering
  \includegraphics[width=\textwidth]{vectors/implementation}
\end{frame}}

\begin{frame}[noframenumbering]{Experimentation}{Materials and methods}
\small
\begin{block}{\small \textbf{Attributes evaluated}}
  \begin{itemize}
    \item Spatial-time accuracy.
    \item Energy consumption.
  \end{itemize}
\end{block}

\begin{block}{\small \textbf{Desktop version}}
  \begin{itemize}
    \item It features different modules of the proposed system (except from HAR module).
    \item It allows the quick evaluation of system performance through different parameter combinations.
    \item It includes a logic \emph{trajectory file} reader that simulates user motion.
  \end{itemize}
\end{block}

\begin{block}{\small \textbf{On-device trials}}
\begin{itemize}
  \item Two Nexus 6 smartphone units.
  \item The smartphones were always carried together with the GPS logger device by a campus student.
  \item No mobile apps other than the developed middleware were executed by the smartphones.
  \item The smartphones were not employed for communication tasks (texting, calls).
\end{itemize}
\end{block}
\end{frame}

\begin{frame}[noframenumbering]{Experimentation}{\emph{Stay Points Detector} module spatial-time accuracy: Results}
\vspace{-0.5cm}
\begin{table}
\centering
\renewcommand{\arraystretch}{0.8}
\resizebox{0.67\textwidth}{!}{
\begin{tabular}{@{}lrrrrrr@{}}
\toprule
\multicolumn{1}{c}{\textbf{Trajectory}} & 
\multicolumn{1}{c}{\textbf{\begin{tabular}[c]{@{}c@{}}Sampling period\\(seconds)\end{tabular}}} & 
\multicolumn{1}{c}{\textbf{\begin{tabular}[c]{@{}c@{}}Live stay\\ points identified\end{tabular}}} & 
\multicolumn{1}{c}{\textbf{\begin{tabular}[c]{@{}c@{}}Average centroid\\ distance (meters)\end{tabular}}} & 
\multicolumn{1}{c}{\textbf{\begin{tabular}[c]{@{}c@{}}Average arrival\\ latency (seconds)\end{tabular}}} & 
\multicolumn{1}{c}{\textbf{\begin{tabular}[c]{@{}c@{}}Average departure\\ latency(seconds)\end{tabular}}} \\ 
\midrule
\multirow{6}{*}{Trajectory 1} 
 & 30 & 12 of 12 & 1.50 &   2.67 & 24.92 \\
 & 60 & 12 of 12 & 3.43 &   \textcolor{red}{\textbf{\emph{-12.33}}} & 17.42 \\
 & 90 & 12 of 12 & 4.04 &   15.17 & 32.42 \\
 & 120 & 12 of 12 & 6.66 &  \textcolor{red}{\textbf{\emph{-7.33}}} & 52.42 \\
 & 150 & 12 of 12 & 9.00 &  25.17 & 79.92 \\
 & 180 & 12 of 12 & 10.88 & 22.67 & 77.42 \\
 \cmidrule(l){1-6}

\multirow{6}{*}{Trajectory 2} 
 & 30 & 16 of 16 & 1.59 & 7.38 & 13.62 \\
 & 60 & 16 of 16 & 4.72 & 20.50 & 34.25 \\
 & 90 & 16 of 16 & 4.42 & 37.38 & 26.75 \\
 & 120 & 16 of 16 & 12.51 & 16.75 & 83.00 \\
 & 150 & 16 of 16 & 15.04 & \textcolor{red}{\textbf{\emph{-0.12}}} & 96.12 \\
 & 180 & 16 of 16 & 15.58 & 65.50 & 71.75 \\
 \cmidrule(l){1-6}

\multirow{6}{*}{Trajectory 3} 
 & 30 & 19 of 19 & 3.39 &  2.26 & 12.16 \\
 & 60 & 19 of 19 & 3.96 &  11.74 & 12.16 \\
 & 90 & 19 of 19 & 8.05 &  40.16 & 29.53 \\
 & 120 & 19 of 19 & 11.16  & 37.00 & 56.37 \\
 & 150 & 19 of 19 & 18.06  & 48.05 & 59.53 \\
 & 180 & 19 of 19 & \textcolor{green}{\textbf{22.52}} & 87.53 & 81.63 \\
 \cmidrule(l){1-6}

\multirow{6}{*}{Trajectory 4} 
 & 30 & 13 of 13 & 0.49 &  16.15 & 2.46 \\
 & 60 & 13 of 13 & 1.05 &  41.54 & 34.77 \\
 & 90 & 13 of 13 & 2.41 &  32.31 & 55.54 \\
 & 120 & 13 of 13 & 4.00 & 32.31 & 71.69 \\
 & 150 & 13 of 13 & 4.78 & 41.54 & 50.92 \\
 & 180 & 13 of 13 & 5.19 & 73.85 & 62.46 \\
 \cmidrule(l){1-6}

\multirow{6}{*}{Trajectory 5} 
 & 30 & 18 of 18 & 1.49 & 0.17 & 13.61 \\
 & 60 & 18 of 18 & 2.31 & 13.50 & 21.94 \\
 & 90 & 18 of 18 & 5.12 & \textcolor{red}{\textbf{\emph{-3.17}}} & 23.61 \\
 & 120 & 18 of 18 & 14.46 & 10.17 & \textcolor{red}{\textbf{\emph{-44.72}}} \\
 & 150 & 18 of 18 & 13.58 & 30.17 & \textcolor{red}{\textbf{\emph{-39.72}}} \\
 & 180 & 18 of 18 & 14.39 & 26.83 & \textcolor{red}{\textbf{\emph{-71.39}}} \\
 \cmidrule(l){1-6}

\multirow{6}{*}{Trajectory 6} 
 & 30 & 75 of 75 & 1.89 & 2.89 & 6.89 \\
 & 60 & \textcolor{blue}{\textbf{76 of 75}} & 3.54 & 8.49 & 24.89 \\
 & 90 & \textcolor{blue}{\textbf{76 of 75}} & 4.67  & 7.29 & 59.69 \\
 & 120 & \textcolor{blue}{\textbf{76 of 75}} & 6.71 & 29.29 & 77.69 \\
 & 150 & 75 of 75 & 9.33 & 42.89 & 113.29 \\
 & 180 & \textcolor{blue}{\textbf{76 of 75}} & 10.29 & 51.69 & 108.89 \\
 \bottomrule 
\end{tabular}
}
\caption{Spatial-time differences in detected stay points per sampling period (ST=stay time). The negative values in the ST difference and the arrival and departure latencies are caused by the combined effect of user mobility and sparse sampling rate on the \emph{StreamedZhen} algorithm, which generates stay points in subtly different coordinates with different time information.}
\end{table}
\end{frame}



\begin{frame}[noframenumbering]{Experimentation}{\emph{Geofencing} module spatial-time accuracy: Results}
\vspace{-0.5cm}
\begin{figure}
  \centering
  \includegraphics[width=0.75\textwidth]{vectors/experiments/exp-2/exp-2-latencies-for-slides}
  \caption{Arrival (left) and departure (right) latencies obtained by the Geofencing  module for each combination of sampling period and window length values. There is a tendency on the results as the shortest window’s sizes produce the shortest arrival latency values.}
\end{figure}
\end{frame}

\begin{frame}[noframenumbering]{Preliminary experiments}{\emph{Geofencing} module spatial-time accuracy: Results}
\vspace{-0.5cm}
\begin{figure}
  \centering
  \includegraphics[width=0.77\textwidth]{vectors/experiments/exp-2/exp-2-departure-distance-for-slides}
  \caption{Departure distance difference obtained by the \emph{Geofencing} module for each combination of sampling period and window size values. On the left, the latency of the window-based geofencing is accounted, while in the right it is ignored. The high speed with which user leaves from stay points allows to identify a tendency on the results, as departures are detected earlier by shorter window sizes.}
\end{figure}
\end{frame}


\begin{frame}[noframenumbering]{Experimentation}{Holistic evaluation: Results}
\small 
\vspace{-0.5cm}
{
  \centering
  \includegraphics[width=0.7\textwidth]{vectors/experiments/exp-4/exp-4-sigmoid-header-top-row}
  \par 
}


\begin{columns}
\begin{column}[T]{0.5\textwidth}

\begin{block}{\small \textbf{Arrival distance difference}}
{
  \centering
  \includegraphics[width=\textwidth]{vectors/experiments/exp-4/exp-4-arrival-distance-for-slides-v2}
  \captionof{figure}{Arrival distance difference detected by the system throughout the different experimental trials. The values are shorter than for departure distance due to the decreasing speed that user describes when arriving to a stay point.}
  \par
}
\end{block}
\end{column}

\begin{column}[T]{0.5\textwidth}
\begin{block}{\small \textbf{Departure distance difference}}
{
  \centering
  \includegraphics[width=\textwidth]{vectors/experiments/exp-4/exp-4-departure-distance-for-slides-v2}
  \captionof{figure}{Departure distance difference detected by the system throughout the different experimental trials. The larger distance differences are caused by the high speed with which user leaves the stay points (mostly using a vehicle as transportation mode).}
  \par
}
\end{block}

\end{column}
\end{columns}

\end{frame}




\begin{frame}[noframenumbering]{Preliminary experiments}{Energy saving expectations of on-device stay points detection}
\small
\vspace{-0.5cm}
\begin{columns}
\begin{column}{0.55\textwidth}
\begin{block}{\small \textbf{Description}}
\begin{itemize}
  \item This experiment explored whether a smartphone could detect stay points by itself, given its energy and computing constraints, and the energy savings of such implementation compared against typical solutions.
  \item Typical solutions implement a Mobile Cloud Computing (MCC) approach on which the smartphone only collects and offloads the processing to external servers.
\end{itemize}
\end{block}
\end{column}

\begin{column}{0.4\textwidth}
\begin{table}
\centering
\renewcommand{\arraystretch}{0.8}
\resizebox{0.95\textwidth}{!}{
\begin{tabular}{lll}
\toprule
\multirow{2}{*}{\textbf{Stay Points Detector}} & \textbf{Time threshold} ($\delta_{time}$): & $45~min$ \\
\cmidrule[0.25pt]{2-3}
 & \textbf{Distance threshold} ($\delta_{distance}$): & $500~m$ \\

\cmidrule[0.25pt]{1-3}
\textbf{Sampling periods}: & \multicolumn{2}{l}{30, 60, 90, 120, 150 seconds} \\
\bottomrule
\end{tabular}
}
\caption{Input parameters for the energy saving expectations of on-device stay points detection experiment.}
\end{table}
\end{column}
\end{columns}

\begin{block}{\small \textbf{Results}}
\vspace{1em}
{\centering
\resizebox{0.9\textwidth}{!}{
\renewcommand{\arraystretch}{0.9}
\begin{tabular}{@{}cccccccc@{}}
\toprule
\textbf{\begin{tabular}[c]{@{}c@{}}Sampling\\period (seconds)\end{tabular}}  &
\textbf{\begin{tabular}[c]{@{}c@{}}Processing\\ strategy\end{tabular}}
& \textbf{\begin{tabular}[c]{@{}c@{}}Location update \\ requests\end{tabular}}
& \textbf{\begin{tabular}[c]{@{}c@{}}GPS-on time\\ (minutes)\end{tabular}}
& \textbf{\begin{tabular}[c]{@{}c@{}}Average acquisition \\  time per fix (seconds)\end{tabular}}
& \textbf{\begin{tabular}[c]{@{}c@{}}Total experiment \\ time (minutes)\end{tabular}}
& \textbf{\begin{tabular}[c]{@{}c@{}}Data sent\\ (bytes)\end{tabular}}
& \textbf{\begin{tabular}[c]{@{}c@{}}Data received\\ (bytes)\end{tabular}} \\ \midrule

\multirow{2}{*}{30}  & On-device   &  12,365 & 1,614 & 7.83  & 7,790 & - & - \\
                          & MCC &  9,324 &   770 & 4.95  & 5,402 & 1,084,901 & 18,796 \\
\cmidrule(l){2-8}
\multirow{2}{*}{60}  & On-device    & 10,851 & 1,219 & 6.74 & 12,028 & - & - \\
                          & MCC &  7,210 &   764 & 6.35 &  7,907 & 838,640 & 14,696 \\
\cmidrule(l){2-8}
\multirow{2}{*}{90}  & On-device    & 7,935 & 1,178 & 8.90 & 13,075 & - & - \\
                          & MCC & 5,626 &   546 & 5.82 &  8,946 & 653,833 & 12,223 \\
\cmidrule(l){2-8}
\multirow{2}{*}{120} & On-device    & 5,246 & 809 & 9.25 & 11,289 & - & - \\
                          & MCC & 4,333 & 387 & 5.35 &  8,931 & 504,012 & 8,838 \\
\cmidrule(l){2-8}
\multirow{2}{*}{150} & On-device    & 5,635 & 933 & 9.93 & 14,998 & - & - \\
                          & MCC & 4,566 & 452 & 5.93 & 11,619 & 530,764 & 10,309 \\
\bottomrule
\end{tabular}
}
\captionof{table}{Summary of experimental results of on-device vs. MCC approach comparison.}
\par }
\end{block}
\end{frame}


\begin{frame}[noframenumbering]{Experimentation}{Energy saving expectations of on-device stay points detection}
\small
\begin{columns}
\begin{column}{0.55\textwidth}
\begin{block}{\small \textbf{Description}}
\begin{itemize}
  \item This experiment explored whether a smartphone could detect stay points by itself, and the energy savings of such implementation with respect of typical Mobile Cloud Computing (MCC) based solutions.
\end{itemize}
\end{block}
\end{column}

\begin{column}{0.4\textwidth}
\begin{table}
\centering
\renewcommand{\arraystretch}{0.8}
\resizebox{0.95\textwidth}{!}{
\begin{tabular}{lll}
\toprule
\multirow{2}{*}{\textbf{Stay Points Detector}} & \textbf{Time threshold} ($\delta_{time}$): & $45~min$ \\
\cmidrule[0.25pt]{2-3}
 & \textbf{Distance threshold} ($\delta_{distance}$): & $500~m$ \\

\cmidrule[0.25pt]{1-3}
\textbf{Sampling periods}: & \multicolumn{2}{l}{30, 60, 90, 120, 150 seconds} \\
\bottomrule
\end{tabular}
}
\caption{Input parameters for the energy saving expectations of on-device stay points detection experiment.}
\label{tab:exp-energy-performance-input-parameters}
\end{table}
\end{column}
\end{columns}

\begin{figure}
  \centering
  \includegraphics[width=\columnwidth]{vectors/local-poi-article/plot-energy-performance-r2-for-slides}
  \caption{Energy performance comparison of on-device vs. MCC sample apps using different GPS sampling periods. Each of the on-device trials last longer than its corresponding remote implementation.}
\end{figure}

\end{frame}



\begin{frame}[noframenumbering]{Experimentation}{Comparison with other solutions}
\begin{table}[]
\centering
\small
\renewcommand{\arraystretch}{1.2}
\begin{tabular}{@{}p{1cm}p{1.8cm}p{1.6cm}p{1.4cm}p{1cm}p{1cm}@{}}
\toprule
\textbf{Work} &
\textbf{Purpose} &
\textbf{Mobility type} &
\textbf{Involved sensors} &
\textbf{Trajectory tracking} &
\textbf{Place learning} \\ 
\midrule

\textbf{SenseLess} &
Location tracking &
Walking, static &
Accelerometer, GPS &
Yes &
No \\

\cmidrule[0.25pt]{1-6}
\textbf{SmartDC} &
Place tracking &
Not specified &
Cellular id, Wi-Fi, GPS&
No &
Yes \\

\cmidrule[0.25pt]{1-6}
Proposed system &
Location and place tracking &
Static, walking, biking, vehicle &
Accelerometer, GPS &
Yes &
Yes \\

\bottomrule
\end{tabular}

\caption{Features comparison of proposed system and representative existing solutions.}
\end{table}
\end{frame}


\begin{frame}[noframenumbering]{Preliminary results}{Comparison with other solutions}
\begin{table}[]
\centering
\small
\tabcolsep=0.12cm
\renewcommand{\arraystretch}{1.2}
\begin{tabular}{@{}p{1cm}p{2cm}p{2.5cm}p{2cm}p{3cm}@{}}
\toprule
\textbf{Work} &
\textbf{Place learning technique} &
\textbf{Sampling adaptation technique} &
\textbf{Energy metric} &
\textbf{Energy result} \\ 
\midrule

\textbf{SenseLess} &
None &
Simple Decision Rule (static, motion) &
Energy profiling of sensors &
41\% consumption (of a fixed $10 s$ GPS sampling) \\

\cmidrule[0.25pt]{1-5}
\textbf{SmartDC} &
Wi-Fi fingerprinting &
Markov Decision Process (time series mobility predictor and energy budget). &
Energy profiling of sensors &
81\% less than a periodic sensing scheme.\\

\cmidrule[0.25pt]{1-5}
Proposed system &
GPS Stream-based SPs detection, Geofencing, time-aware visits analysis. &
Cognitive controller (sigmoid sampling) &
Battery lifetime, amount of location updates &
{
Simulation: 14\%, 42\% location updates (of fixed $10 s$, $30 s$ GPS sampling).

On device deployment: 53 hours battery life increase, 77\% location updates (of a fixed $30 s$ GPS sampling), 
}\\

\bottomrule
\end{tabular}

\caption{Results comparison of proposed system and representative existing solutions.}
\end{table}
\end{frame}
