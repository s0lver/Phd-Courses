\section{Objectives}
\subsection{Objectives}
\begin{frame}{Objectives}
\small
\begin{block}{\small \textbf{Main objective}}
\begin{itemize}
  \item To reduce the energy consumption of mobile sensing apps, which perform continuous sensor sampling, through self-adapting power-aware policies generated from context information obtained from sensors data.
\end{itemize}
\end{block}

\begin{block}{\small \textbf{Particular objectives}}
\begin{itemize}
  \item To detect mobility patterns from context information obtained from an inertial sensor (accelerometer) and location provider (GPS).
  \item To generate an accurate representation of detected patterns for summarizing user mobility.
  \item To dynamically adapt GPS sampling rate by means of a cognitive controller that employs the learned mobility representation and accuracy requirements for implementing power-aware sampling policies.
  \item To ease the development of mobile sensing applications that require user location tracking, i.e., LBSs and MBSs, isolating the complexity of sensors access and the associated efficient energy management.
\end{itemize}
\end{block}
\end{frame}

\subsection{Contributions}
\begin{frame}{Contributions}
\small
\begin{block}{\small \textbf{Contributions}}
\begin{itemize}
  \item An on-device mobility patterns detector that works with streams of raw data collected by smartphone's sensors (GPS and accelerometer).
  \item An on-device mobility analyzer that incrementally builds a model of user mobility from the detected mobility patterns.
  \item A cognitive controller inspired on CDSs that, based on the mobility information learned, dynamically adapts GPS sampling rate through power-aware policies. 
  \item A middleware with the previous modules embedded for easing the development of LBSs and MBSs for the Android mobile platform.
\end{itemize}
\end{block}
\end{frame}