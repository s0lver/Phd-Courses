%!TEX program = lualatex
\documentclass[8pt,compress,xcolor={table,dvipsnames}]{beamer}

\usepackage{polyglossia}
\setmainlanguage{english}

\usetheme{hsrm}
\setbeameroption{show notes}
\usepackage{helvet}
\usepackage{tabularx,ragged2e}
\usepackage{booktabs}
% \usepackage{dtklogos}
\usepackage{tikz}
\usetikzlibrary{mindmap,backgrounds}
\usepackage{arydshln}
\usepackage{chngpage}
\usepackage{natbib}
\usepackage{multicol}
\usepackage[caption=false]{subfig}
\usepackage[absolute,overlay]{textpos}

\definecolor{colorAzulBienBonito}{RGB}{0,200,255}
\newcommand{\markdone}{\textcolor{PineGreen}{$\bullet$}}
\newcommand{\markonprogress}{\textcolor{colorAzulBienBonito}{$\bullet$}}
\newcommand{\markoff}{\textcolor{PineGreen}{$\circ$}}

%\usepackage[natbib=true, bibstyle=authoryear, citestyle=authoryear-comp, backend=bibtex]{biblatex}
\setbeamersize{text margin left=5mm,text margin right=5mm}
\graphicspath{ {../../../resources/images/}}

\title{Smart usage of context information for the analysis, design, and generation of power-aware policies for mobile sensing apps}

\subtitle{Doctoral seminar}

\author{Rafael Pérez Torres}

\institute{Center for Research and Advanced Studies of the National Polytechnic Institute\\LTI {\Medium Cinvestav}}

\date{September 2017}

\begin{document}

\maketitle

\begin{frame}{Structure}{}
  \tableofcontents[hideallsubsections]
\end{frame}

%!TEX root = ../slides.tex
\section{Introduction}
\subsection{Motivation}
\begin{frame}{Research background}{Motivation}
\begin{figure}[tb]
  \centering
  \includegraphics[width=0.4\textwidth]{vectors/smartphone-dimensions-v2}
  \caption{The advances in the communication, computing and sensing dimensions of mobile devices contribute to their acceptance by society~\cite{Islam2014}.}  
\end{figure}

\begin{block}{\small \textbf{Motivation}}
{
  \small
\begin{itemize}
  \item The sensing dimension enables \emph{context-awareness} in mobile devices, such as the smartphone.
  \item Battery advances are slower than those of other smartphone components~\cite{Kjaergaard2012}, growing 5-10\% yearly~\cite{Ma2012,Evarts2015}, a critical issue for the \textbf{mobile sensing applications}.
  % \item Battery advances are slower than those of other smartphone components~\cite{Kjaergaard2012}, growing 5-10\% yearly~\cite{Ma2012,Evarts2015}.
  % \item The energy constraint is critical for the continuous access to sensors required by \textbf{mobile sensing applications}. 
  \item Scientific efforts have been done for achieving the energy efficiency of the GPS location provider.
  \item The understanding of mobility could augment the location-awareness of the smartphone for many purposes, such as energy savings and the development of Mobility Based Services (MBSs).
  % \item As a high level of abstraction, mobility can be characterized as a sequence of frequently visited places (stay points).
\end{itemize}
}
\end{block}
\end{frame}

% \begin{frame}{Research background}{Motivation}
% \begin{block}{\small \textbf{Motivation}}
% {
%   \small
%    \begin{itemize}
%      \item For the sensing dimension, scientific efforts have been done for achieving the energy efficiency of the GPS location provider.
%      \item The understanding of mobility could augment the location-awareness of the smartphone for many purposes, such as energy savings and the development of Mobility Based Services (MBSs).
%      \item As a high level of abstraction, mobility can be characterized as a sequence of frequently visited places (a.k.a. stay points).
% \end{itemize}
% }
% \end{block}
% \end{frame}

\subsection{Research background}
%\subsection{Problem statement}
\begin{frame}{Research background}{Problem statement}
\small
\vspace{-0.5cm}
\begin{itemize}
  \item The understanding of mobility is possible at different spatial-temporal scales:
\end{itemize}

\begin{exampleblock}{\small \textbf{Fine-grain mobility patterns identification}}
 \begin{itemize}
    \item They refer to the transportation mode employed by user when moving between stay points.
    \item Given a set of values $\mathcal{V} = v_{acc~1},v_{acc~2},\ldots,v_{acc~n}$ obtained from accelerometer in the time interval $[t_1,t_2]$, identify fine-grain mobility information:
\begin{equation*}
  \text{\textbf{FineGrainMobilityIdentifier}}(\mathcal{V}) \rightarrow p_S \in \{ \text{static, walking, biking, vehicle} \}
\end{equation*}
with each $v_{acc~i} \in \mathcal{V}$ composed as $\langle acc_x,acc_y,acc_z,t \rangle$.
  \end{itemize} 
\end{exampleblock}

\begin{exampleblock}{\small \textbf{Coarse-grain mobility patterns identification}}
 \begin{itemize}
   \item They refer to motion at a large spatial scale related to user visiting stay points.
   \item \sloppy Given a set of values $\mathcal{V} = v_{gps~1},v_{gps~2},\ldots,v_{gps~n}$ obtained from GPS location provider in time interval $[t_1,t_2]$, identify coarse-grain mobility information:
\begin{equation*}
    \text{\textbf{CoarseGrainMobilityIdentifier}}(\mathcal{V}) \rightarrow p_S \in \{ \text{new stay point, arrival, departure} \}
\end{equation*}
with each $v_{gps~i} \in \mathcal{V}$ composed associated $\langle lat, lon, t \rangle$.
 \end{itemize}
\end{exampleblock}
\end{frame}


\begin{frame}{Research background}{Problem statement}
\small

\begin{exampleblock}{\small \textbf{Sensors sampling adaptation}}
\begin{itemize}
\item Given a set of coarse and fine-grain mobility patterns $\mathcal{P} = \{ p_{S_1}, p_{S_2}, \ldots, p_{S_n} \}$, and accuracy requirements of mobile app $req_{accuracy}$, implement a sampling policy for the adaptive duty cycling of sensors while reducing energy consumption:
\begin{equation*}
  \text{\textbf{PolicyGeneration}}( \mathcal{P}, req_{accuracy}) \longrightarrow{} \mathcal{S}_{conf}
\end{equation*}
where $\mathcal{S}_{conf} \rightarrow s, \mathcal{T}_{real}$ represents the sampling $\mathcal{T}_{real}$ that must be implemented for sensor $s$.
The $req_{accuracy}$ refers to the granularity of GPS sampling.
\end{itemize}
\end{exampleblock}

\begin{figure}[tb]
  \centering
  \includegraphics[width=0.7\textwidth]{vectors/problems-incorporation-v2}
  \caption{Interaction between problems.}
\end{figure}
\end{frame}

% \subsection{Hypothesis}
\begin{frame}{Research background}{Hypothesis}
\small
\begin{block}{\small \textbf{Hypothesis}}
\renewcommand{\baselinestretch}{1.4}
\begin{itemize}
  \item The energy consumption of continuous and extended location tracking could be reduced by means of a cognitive dynamic system that learns an expanded spatial-time model from mobility events detected from sensors data and that employs such model in a cognitive controller for dynamically adapting GPS sampling rate through sampling policies tailored to current mobility state.
\end{itemize}
\end{block}
\end{frame}


% \subsection{Objectives}
\begin{frame}{Research background}{Objectives}
\small
\begin{block}{\small \textbf{Main objective}}
\begin{itemize}
  \item To reduce the energy consumption of mobile sensing apps, which perform continuous sensor sampling, through self-adapting power-aware policies generated from context information obtained from sensors data.
\end{itemize}
\end{block}

\begin{block}{\small \textbf{Particular objectives}}
\begin{itemize}
  \item To detect mobility patterns from context information obtained from an inertial sensor (accelerometer) and location provider (GPS).
  \item To generate an accurate representation of detected patterns for summarizing user mobility.
  \item To dynamically adapt GPS sampling rate by means of a cognitive controller that employs the learned mobility representation and accuracy requirements for implementing power-aware sampling policies.
  \item To ease the development of mobile sensing applications that require user location tracking, i.e., LBSs and MBSs, isolating the complexity of sensors access and the associated efficient energy management.
\end{itemize}
\end{block}
\end{frame}



% \subsection{Methodology}
\begin{frame}{Research background}{Methodology}
\small
\begin{block}{\small \textbf{Methodology}}
\begin{enumerate}
  \item \textbf{Revision of state of the art power-aware sensing techniques.}
  \item \textbf{Formal definition and selection of mobility patterns to be identified.}
  \item \textbf{Research on algorithms for detecting mobility patterns.}
  \item \textbf{Design of the \emph{Mobility Events Detector}.}
  \item \textbf{Design of adaptive policies for energy efficient usage of sensors.}
  \item \textbf{Design of the Cognitive Controller.}
  \item \textbf{Development of a middleware involving the \emph{Mobility Events Detector} and the Cognitive Controller for the Android platform.}
  \item Experimentation in terms of spatial-time accuracy and energy efficiency.
\end{enumerate}
\end{block}
\end{frame}


\section{Problem statement}
\subsection{Problems definition}
\begin{frame}{Problem statement}
\small
\begin{itemize}
  \item The understanding of mobility is possible at different spatial-temporal scales:
\end{itemize}

\begin{exampleblock}{\small \textbf{Fine-grain mobility patterns identification}}
 \begin{itemize}
    \item They refer to the transportation mode employed by user when moving between stay points.
    \item Given a set of values $\mathcal{V} = \left \{ v_{acc~1},v_{acc~2},\ldots,v_{acc~n} \right \}$ obtained from accelerometer in the time interval $[t_1,t_2]$, identify fine-grain mobility information:
\begin{equation*}
  \text{\textbf{FineGrainMobilityIdentifier}}(\mathcal{V}) \rightarrow p_S \in \{ \text{static, walking, biking, vehicle} \}
\end{equation*}
with each $v_{acc~i} \in \mathcal{V}$ composed as $\langle acc_x,acc_y,acc_z,t \rangle$.
  \end{itemize} 
\end{exampleblock}

\begin{exampleblock}{\small \textbf{Coarse-grain mobility patterns identification}}
 \begin{itemize}
   \item They refer to motion at a large spatial scale related to user visiting stay points.
   \item \sloppy Given a set of values $\mathcal{V} = \left \{ v_{gps~1},v_{gps~2},\ldots,v_{gps~n} \right \}$ obtained from GPS location provider in time interval $[t_1,t_2]$, identify coarse-grain mobility information:
\begin{equation*}
    \text{\textbf{CoarseGrainMobilityIdentifier}}(\mathcal{V}) \rightarrow p_S \in \{ \text{new stay point, arrival, departure} \}
\end{equation*}
with each $v_{gps~i} \in \mathcal{V}$ composed associated $\langle lat, lon, t \rangle$.
 \end{itemize}
\end{exampleblock}
\end{frame}

\subsection{Hypothesis}
\begin{frame}{Hypothesis}
\small
\begin{block}{\small \textbf{Hypothesis}}
\renewcommand{\baselinestretch}{1.4}
\begin{itemize}
  \item The energy consumption of continuous and extended location tracking could be reduced by means of a cognitive dynamic system that learns an expanded spatial-time model from mobility events detected from sensors data and that employs such model in a cognitive controller for dynamically adapting GPS sampling rate through sampling policies tailored to current mobility state.
\end{itemize}
\end{block}
\end{frame}

\section{Objectives}
\subsection{Objectives}
\begin{frame}{Objectives}
\small
\begin{block}{\small \textbf{Main objective}}
\begin{itemize}
  \item To reduce the energy consumption of mobile sensing apps, which perform continuous sensor sampling, through self-adapting power-aware policies generated from context information obtained from sensors data.
\end{itemize}
\end{block}

\begin{block}{\small \textbf{Particular objectives}}
\begin{itemize}
  \item To detect mobility patterns from context information obtained from an inertial sensor (accelerometer) and location provider (GPS).
  \item To generate an accurate representation of detected patterns for summarizing user mobility.
  \item To dynamically adapt GPS sampling rate by means of a cognitive controller that employs the learned mobility representation and accuracy requirements for implementing power-aware sampling policies.
  \item To ease the development of mobile sensing applications that require user location tracking, i.e., LBSs and MBSs, isolating the complexity of sensors access and the associated efficient energy management.
\end{itemize}
\end{block}
\end{frame}

\subsection{Contributions}
\begin{frame}{Contributions}
\small
\begin{block}{\small \textbf{Contributions}}
\begin{itemize}
  \item An on-device mobility patterns detector that works with streams of raw data collected by smartphone's sensors (GPS and accelerometer).
  \item An on-device mobility analyzer that incrementally builds a model of user mobility from the detected mobility patterns.
  \item A cognitive controller inspired on CDSs that, based on the mobility information learned, dynamically adapts GPS sampling rate through power-aware policies. 
  \item A middleware with the previous modules embedded for easing the development of LBSs and MBSs for the Android mobile platform.
\end{itemize}
\end{block}
\end{frame}

\section{Methodology}
\subsection{Methodology}
\begin{frame}{Methodology}
\small
\begin{block}{\small \textbf{Steps}}
\begin{enumerate}
  \item Revision of state of the art power-aware sensing techniques.
  \item Formal definition and selection of mobility patterns to be identified.
  \item Research on algorithms for detecting mobility patterns.
  \item Design of the \emph{Mobility Events Detector}.
  \item Design of adaptive policies for energy efficient usage of sensors.
  \item Design of the Cognitive Controller.
  \item Development of a middleware involving the \emph{Mobility Events Detector} and the Cognitive Controller for the Android platform.
  \item Experimentation in terms of spatial-time accuracy and energy efficiency.
\end{enumerate}
\end{block}
\end{frame}




\section{Solution}
\subsection{Solution's overview}
{
\begin{frame}[plain]
  \begin{textblock*}{8.5cm}(4.3cm,8.3cm)
  \small
  \textbf{Architecture of proposed system solution. The global perception-action cycle, the different memory types, and the internal feedback loops are observed.}
  \end{textblock*}

  \begin{textblock*}{1cm}(12cm,9.3cm)
  \scriptsize
  \insertframenumber~/~\inserttotalframenumber
  \end{textblock*}

  \centering
  \includegraphics[width=0.98\textwidth]{vectors/inspired-cds-solution-for-slides}
  %\captionof{figure}{Architecture of proposed system solution. The global perception-action cycle and the internal loops are observed.}
\end{frame}}




% \bibliographystyle{unsrtnat}
% \begin{frame}{References}
% \frame[allowframebreaks]{ 
% 		\bibliography{../../../resources/references/library}
% }
% \end{frame}

\begin{frame}[allowframebreaks,noframenumbering]
        \frametitle{References}
%\bibliographystyle{plain}
{
\tiny{}
\bibliographystyle{unsrt}
\bibliography{../../../resources/references/library}
}
\end{frame}

\begin{frame}[noframenumbering]{Detailed schedule}
\begin{table}[]
\centering
\resizebox{0.95\textwidth}{!}{%
\begin{tabular}{clccccccc}
   &                                                                               & 2014 & \multicolumn{3}{c}{2015} & \multicolumn{3}{c}{2016}  \\
   \cline{3-9}
\multicolumn{2}{l}{\small{Work: \markdone~Done, \markonprogress~In progress, \markoff~To be done}} & 3rd & 1st & 2nd & 3rd & 1st & 2nd & 3rd \\
   \toprule
\multicolumn{2}{c}{\textsc{Step I}}                                                &  &  &  &  &  &  &  \\
1  & State-of-art reading                                                          & \markdone & \markdone &  &  &  &  &  \\
2  & State-of-art works categorization                                             &  & \markdone & \markdone  &  &  &  &  \\
3  & Documentation of information found (committee request)                        &  &  & \markdone &  &  &  &  \vspace{1em}\\

\multicolumn{2}{c}{\textsc{Step II}}                                               &  &  &  &  &  &  &  \\
4  & \makecell[l]{Development of a mobile app for accelerometer and location \\data collection}    &  &  & \markdone & \markdone  &  &  &  \\
5  & Analysis of data                                                              &  &  &  & \markdone  &  &  &  \\
6  & Formal definition of mobility pattern                                         &  &  &  & \markdone &  &  &  \\
7  & Selection of mobility patterns                                                &  &  &  & \markdone & \markdone &  &  \vspace{1em}\\

\multicolumn{2}{c}{\textsc{Step III}}                                              &  &  &  &  &  &  &  \\
8  & Research on classification algorithms for mobility patterns                   &  &  &  & \markdone & \markdone &  &  \\
9  & Definition of metrics for evaluating algorithms                               &  &  &  &  & \markdone &  &  \\
10 & Implementation of algorithms in mobile platform                               &  &  &  &  & \markdone & \markdone  &  \\
11 & Selection of best algorithms according to metrics                             &  &  &  &  &  & \markdone &  \vspace{1em}\\

\multicolumn{2}{c}{\textsc{Step IV}}                                               &  &  &  &  &  &  &  \\
12 & Definition and modeling of parameters needed by the \emph{Mobility Events Detector}                       &  &  &  & \markdone & \markdone & \markdone & \markdone \\
13 & Development of the \emph{Mobility Events Detector}.  &  &  &  & \markdone & \markdone & \markdone & \markdone \\
\bottomrule
\end{tabular}%
}
\caption{Schedule of activities (each column represents a four months period)}
\end{table}
\end{frame}

\begin{frame}[noframenumbering]{Detailed schedule}
\begin{table}[]
\centering
\resizebox{0.95\textwidth}{!}{%
\begin{tabular}{clccccccc}
   &                                                                               & \multicolumn{3}{c}{2016} & \multicolumn{3}{c}{2017} & 2018 \\
   \cline{3-9}
\multicolumn{2}{l}{\small{Work: \markdone~Done, \markonprogress~In progress, \markoff~To be done}}   & 1st & 2nd & 3rd & 1st & 2nd & 3rd & 1st \\
   \toprule
\multicolumn{2}{c}{\textsc{Step V}}                                                &  &  &  &  &  &  & \\
14  & Formal definition of policy                                                  &  &  & \markdone &  &  &  & \\
15  & \makecell[l]{Research and evaluation of techniques for\\generation and adaption of policies}&  &  & \markdone & \markdone  &  &  & \\
16  & Design and execution of experiments applied to use cases                     &  &  & \markdone & \markdone  &  &  & \\
17  & Selection of policies                                                        &  &  &  & \markdone  &  &  & \vspace{1em} \\

\multicolumn{2}{c}{\textsc{Step VI}}                                               &  &  &  &  &  &  & \\
18  & Definition and modeling of Cognitive Controller parameters                                    &  &  &  &  & \markdone  &  & \\
19  & Development of the Cognitive controller                                                          &  &  &  &  & \markdone  &  & \vspace{1em} \\

\multicolumn{2}{c}{\textsc{Step VII}}                                               &  &  &  &  &  &  & \\
20  & Analysis of components into software abstractions                             &  &  &  &  & \markdone  &  & \\
21  & Research on Android API for specialized components                            &  &  &  &  & \markdone  &  & \\
22  & Development of middleware                                                     &  &  &  &  & \markdone  &  & \vspace{1em}\\

\multicolumn{2}{c}{\textsc{Step VIII}}                                              &  &  &  &  &  &  & \\
23  & \makecell[l]{Definition of experiments aimed at accuracy\\and energy consumption metrics}    &  &  &  &  &  & \markdone & \\
24  & Development of experimental sample mobile apps                                &  &  &  &  &  & \markonprogress & \\
25  & Experiments execution                                                         &  &  &  &  &  & \markonprogress & \markoff  \\
26  & Final results analysis                                                              &  &  &  &  &  &  & \markoff \\
\bottomrule
\end{tabular}%
}
\caption{Schedule of activities (each column represents a four months period)}
\end{table}
\end{frame}

\begin{frame}[noframenumbering]{Detailed schedule}
\begin{table}
\centering
\resizebox{\textwidth}{!}{%
\begin{tabular}{clcccccccccccc}
   &                                                                               & 2014 & \multicolumn{3}{c}{2015} & \multicolumn{3}{c}{2016} & \multicolumn{3}{c}{2017} & \multicolumn{2}{c}{2018} \\
   \cline{3-14}
 \multicolumn{2}{l}{\small{Work: \markdone~Done, \markonprogress~In progress, \markoff~To be done}} & 3rd & 1st & 2nd & 3rd & 1st & 2nd & 3rd & 1st & 2nd & 3rd & 1st & 2nd\\
   \toprule
\multicolumn{2}{c}{\textsc{Required tasks}}                                        &  &  &  &  &  &  &  &  &  &  &  & \\
A   & Related courses                                                              & \markdone  & \markdone  & \markdone &  &  &  &  &  &  &  &  & \\
B   & Research articles submission                                                 &  &  &  & \markdone &  & \markdone &  &  &  &  & \markonprogress & \\
C   & Predoctoral exam preparation                                                 &  &  &  &  &  &  &  &  & \markdone &  &  & \\
D   & Thesis writing                                                               & \markdone  &  &  & \markdone &  &  & \markdone &  &  & \markonprogress & \markoff & \markoff \\

\bottomrule
\end{tabular}%
}
\caption{Schedule of required activities}
\end{table}
\end{frame}


\begin{frame}[noframenumbering]{Layered perceptual memory}{Short and long-term memory information}
\vspace{-0.25cm}
\small
\begin{block}{\small \textbf{Layered perceptual memory}}
\begin{itemize}
    \item Short-term memory information: current (observed) mobility status.
    \item Long-term memory information: the Expanded Spatial-Time model (STM).
\end{itemize}
\end{block}

\begin{block}{\small \textbf{Expanded Spatial-Time model}}
\begin{itemize}
  \item The highest level of mobility information held by the system.
\end{itemize}
{
  \centering
  \includegraphics[width=0.75\textwidth]{vectors/stm-slides}
  \captionof{figure}{A conceptual representation of the STM's structure.}
\par }
\end{block}
\end{frame}

\begin{frame}[noframenumbering]{Layered perceptual memory}{Expanded Spatial-Time model (STM)}
\small
\begin{block}{\small \textbf{Generation of the STM}}
\begin{itemize}
    \item Incrementally built with the coarse-grain mobility events detected by the \emph{Mobility Events Detector}.
\end{itemize}
{
  \centering
  \includegraphics[width=\textwidth]{vectors/event-driven-memory-generation-for-slides}
  \captionof{figure}{A conceptual representation of the steps for generating the STM from raw sensors data.}
  \par
}
\end{block}
\end{frame}


\begin{frame}[noframenumbering]{Cognitive controller (CC)}{Description}
\small
\vspace{-0.5cm}
\begin{columns}
\begin{column}[T]{0.5\textwidth}
\begin{block}{\small \textbf{Goals}}
  \begin{itemize}
      \item To reduce the energy consumption of location tracking by relying on PRM's estimations.
      \item To reduce the system uncertainty about current user mobility.
  \end{itemize}
\end{block}
\end{column}

\begin{column}[T]{0.5\textwidth}
\begin{block}{\small \textbf{Possible cognitive actions}}
  \begin{itemize}
    \item \textbf{Exploitation policies}: When system uncertainty is low for saving energy purposes.
    \item \textbf{Exploration policies}: When system uncertainty is high for recovering for accuracy loss.
  \end{itemize}
\end{block}
\end{column}
\end{columns}

\begin{figure}
  \centering
  \includegraphics[width=0.65\textwidth]{vectors/cognitive-controller-architecture}
  \caption{A generic cognitive controller architecture}
\end{figure}
\end{frame}

\begin{frame}[noframenumbering]{Cognitive controller}{Policies tailored for user mobility}
\small
\begin{block}{\small \textbf{Stay point mode}}
  \begin{itemize}
      \item A sampling based on the sigmoid function $sig(x) = \frac{1}{1+e^{-\alpha x}}$ as a model for the mobility phase transitions.
      \item Higher sampling rate on arrival and departure, when the user is more likely to move, and slower at the middle of a visit.
  \end{itemize}
\end{block}

\begin{columns}
\begin{column}{0.45\textwidth}
\begin{figure}
  \centering
  \includegraphics[width=0.99\linewidth]{vectors/sigmoid-segmentation-for-slides}
  \caption{Approximation of the sigmoid through straight segments.}
\end{figure}
\end{column}

\begin{column}{0.55\textwidth}
\begin{figure}
  \centering
  \includegraphics[width=0.99\linewidth]{vectors/sigmoid-driven-sampling-policy-for-slides}
  \caption{A snapshot of the process for producing a sigmoid sampling.}
\end{figure}
\end{column}
\end{columns}
\end{frame}

\begin{frame}[noframenumbering]{Preliminary experimentation}{\emph{Stay Points Detector} module spatial-time accuracy}
\small

\begin{columns}
\begin{column}[T]{0.55\textwidth}

\begin{block}{\small \textbf{Description}}
\begin{itemize}
  \item This experiment evaluates the spatial-time accuracy of the \emph{Stay Points Detector} module under different GPS sampling rates in terms of centroid distances and latencies.
\end{itemize}
\end{block}

\end{column}
\begin{column}[T]{0.45\textwidth}
\begin{table}
\centering
\renewcommand{\arraystretch}{0.6}
\resizebox{0.9\textwidth}{!}{%
\begin{tabular}{lll}
\toprule
\multirow{2}{*}{\textbf{Stay Points Detector}} & \textbf{Time threshold} ($\delta_{time}$): & $45~min$ \\
\cmidrule[0.25pt]{2-3}
 & \textbf{Distance threshold} ($\delta_{distance}$): & $500~m$ \\

\cmidrule[0.25pt]{1-3}
\textbf{Sampling periods}: & \multicolumn{2}{l}{30, 60, 90, 120, 150, 180 seconds.} \\

\cmidrule[0.25pt]{1-3}
\textbf{Trajectories}: & \multicolumn{2}{l}{All ground truth trajectories.} \\
\bottomrule
\end{tabular}
}
\caption{Input parameters for the spatial-time accuracy of stay points experiment.}
\end{table}
\end{column}
\end{columns}

\vspace{-0.5cm}
\begin{block}{\small \textbf{Results}}
\begin{columns}
\begin{column}{0.65\textwidth}
\begin{figure}
  \centering
  \includegraphics[width=0.99\textwidth]{vectors/exp-1-centroid-distance}
\end{figure}
\end{column}
\begin{column}{0.3\textwidth}
\captionof{figure}{The impact of different sampling periods on the centroid distance of identified stay points in each trajectory. A maximum centroid distance of $22.52~m$ is identified when employing the 180 seconds sampling period.}
\end{column}
\end{columns}

\end{block}
\end{frame}


\begin{frame}[noframenumbering]{Preliminary experimentation}{\emph{Geofencing} module spatial-time accuracy: Results}
\vspace{-0.5cm}
\begin{figure}
  \centering
  \includegraphics[width=0.78\textwidth]{vectors/exp-2-visits-for-slides}
  \caption{Visits missed by the \emph{Geofencing} module for each combination of sampling period and window size values. The largest amount is obtained for the \emph{Trajectory 6}, given its length (more than 30 days). Nevertheless, they do not account for a considerable time in overall trajectories.}
\end{figure}
\end{frame}


\begin{frame}[noframenumbering]{Preliminary experimentation}{Holistic evaluation: Results}
\small 
\vspace{-0.5cm}
{
  \centering
  \includegraphics[width=0.7\textwidth]{vectors/exp-4-sigmoid-header-top-row}
  \par 
}

\begin{columns}
\begin{column}[T]{0.48\textwidth}
\begin{block}{\small \textbf{Arrival latency}}
{
  \centering
  \includegraphics[width=\textwidth]{vectors/exp-4-arrival-latency-for-slides-v2}
  \captionof{figure}{Arrival latency observed by the platform in experimental trials. The largest average value is below 65 seconds, explained by the fact that 2 location updates must be collected by the \emph{Geofencing} module before identifying an arrival event.}
  \par 
}
\end{block}
\end{column}

\begin{column}[T]{0.52\textwidth}
\begin{block}{\small \textbf{Departure latency}}
{
  \centering
  \includegraphics[width=\textwidth]{vectors/exp-4-departure-latency-for-slides-v2}
  \captionof{figure}{Departure latency observed by the platform across performed experimental trials. The latencies are within 65 and 165 seconds, which is aligned with the different values specified to the CC for its sigmoid-driven sampling.}
  \par
}
\end{block}
\end{column}
\end{columns}
\end{frame}

\begin{frame}[noframenumbering]{Preliminary experimentation}{Holistic evaluation: Results}
\small
\vspace{-0.5cm}
{
  \centering
  \includegraphics[width=0.7\textwidth]{vectors/exp-4-sigmoid-header-top-row}
  \par 
}

\begin{columns}
\begin{column}[T]{0.5\textwidth}

\begin{block}{\small \textbf{Trajectory distance difference}}
{
  \centering
  \includegraphics[width=\textwidth]{vectors/exp-4-on-trajectory-distance-for-slides-v2}
  \captionof{figure}{The average distance of equivalent trajectory segments during experimental trials. The values are enclosed within $2.5~m$ and $8.5~m$, with the 30 seconds sampling obtaining the lowest values in each trial.}
  \par
}
\end{block}

\end{column}

\begin{column}[T]{0.5\textwidth}
\begin{block}{\small \textbf{Overall reduction of location update requests}}
{
  \centering
  \includegraphics[width=\textwidth]{vectors/exp-4-location-requests-for-slides-v2}
  \captionof{figure}{The proportion of location update requests employed by each experimental trial with respect to the corresponding $1~Hz$ ground truth trajectory. All of the parameter combinations outperform the 30 seconds sampling period, which provides a rough estimation of the energy savings that the system could achieve in on-device implementations.}
  \par
}
\end{block}
\end{column}
\end{columns}
\end{frame}

\begin{frame}[noframenumbering]{Preliminary experimentation}{Energy saving expectations of on-device stay points detection}
\small
% \vspace{-0.5cm}
\begin{columns}
\begin{column}{0.55\textwidth}
\begin{block}{\small \textbf{Description}}
\begin{itemize}
  \item This experiment explored whether a smartphone could detect stay points by itself, and the energy savings of such implementation with respect of typical Mobile Cloud Computing (MCC) based solutions.
  % \item Typical solutions implement a Mobile Cloud Computing (MCC) approach on which the smartphone only collects and offloads the processing to external servers.
\end{itemize}
\end{block}
\end{column}

\begin{column}{0.4\textwidth}
\begin{table}
\centering
\renewcommand{\arraystretch}{0.8}
\resizebox{0.95\textwidth}{!}{%
\begin{tabular}{lll}
\toprule
\multirow{2}{*}{\textbf{Stay Points Detector}} & \textbf{Time threshold} ($\delta_{time}$): & $45~min$ \\
\cmidrule[0.25pt]{2-3}
 & \textbf{Distance threshold} ($\delta_{distance}$): & $500~m$ \\

\cmidrule[0.25pt]{1-3}
\textbf{Sampling periods}: & \multicolumn{2}{l}{30, 60, 90, 120, 150 seconds} \\
\bottomrule
\end{tabular}
}
\caption{Input parameters for the energy saving expectations of on-device stay points detection experiment.}
\end{table}
\end{column}
\end{columns}

\begin{figure}
  \centering
  \includegraphics[width=\columnwidth]{vectors/plot-energy-performance-r2-for-slides}
  \caption{Energy performance comparison of on-device vs. MCC sample apps using different GPS sampling periods. Each of the on-device trials last longer than its corresponding remote implementation.}
\end{figure}

\end{frame}


\begin{frame}[noframenumbering]{Preliminary experimentation}{Energy consumption of fixed-sampling periods: Results}
\vspace{-0.4cm}
{
  \centering
  \includegraphics[width=0.85\textwidth]{vectors/exp-6-energy-burnout}
  \captionof{figure}{Energy performance of a fixed 30 seconds sampling versus a basic sampling adaptation consisting in a 30 seconds sampling in trajectory mode and a slower sampling rate during stay point mode. The separation between the lines in each plot starts after the system learns the stay points with the largest weight in user mobility (home and work places).}
  \par
}
\end{frame}


\begin{frame}[noframenumbering]{Preliminary experimentation}{Energy consumption of fixed-sampling periods: Results}
\vspace{-0.4cm}
{
  \centering
  %\includegraphics[width=0.85\textwidth]{vectors/exp-6-energy-burnout}
  %\captionof{figure}{Energy performance of a fixed 30 seconds sampling versus a basic sampling adaptation consisting in a 30 seconds sampling in trajectory mode and a slower sampling rate during stay point mode. The separation between the lines in each plot starts after the system learns the stay points with the largest weight in user mobility (home and work places).}
  \includegraphics[width=0.8\textwidth]{vectors/energy-results-v2}
  \captionof{figure}{Energy performance obtained by the CDS along the mobility described by user (trial corresponding to the 30 seconds in trajectory and 60 seconds in stay point sampling).}
  \par
}
\end{frame}

\begin{frame}[noframenumbering]{Preliminary experimentation}{Energy consumption of fixed-sampling periods: Results}
\begin{figure}
  % \includegraphics[width=0.8\textwidth]{vectors/exp-6-map-trj-4-adaptive-sampling-60-sp}
  \includegraphics[width=0.85\textwidth]{vectors/stay-points-as-3d-v2}
  \caption{The stay points and visits identified by the system, as well as the mobility summary obtained from such information (trial corresponding to the 30 seconds in trajectory and 60 seconds in stay point sampling).}
%  \caption{The information autonomously learned by the STM during the trial corresponding to the 30 seconds in trajectory and 60 seconds in stay point sampling scheme. The height of cylinders corresponds with the stay time during each stay point visit.}
\end{figure}
\end{frame}


\end{document}
