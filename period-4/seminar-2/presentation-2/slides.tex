%!TEX program = lualatex
%\documentclass[aspectratio=169,compre%ss,9pt]{beamer}
\documentclass[compress,9pt,xcolor={dvipsnames,table}]{beamer}
\usepackage[T1]{fontenc}
%\usepackage{lmodern}
\usepackage{textcomp}
\usepackage{algorithm}
\usepackage{algorithmic}
\usepackage{tcolorbox}
\usepackage{fontspec}

\usepackage{polyglossia}
\setmainlanguage{english}
\usepackage{tabularx,ragged2e}
\usepackage{booktabs}
\usepackage{dtklogos}
\usepackage{arydshln}
\usepackage{chngpage}

\usepackage{booktabs}
\usepackage{tikz}
\def\checkmark{\tikz\fill[scale=0.4](0,.35) -- (.25,0) -- (1,.7) -- (.25,.15) -- cycle;}
\usepackage{array}
\newcolumntype{L}[1]{>{\raggedright\let\newline\\\arraybackslash\hspace{0pt}}m{#1}}
\newcolumntype{C}[1]{>{\centering\let\newline\\\arraybackslash\hspace{0pt}}m{#1}}
\newcolumntype{R}[1]{>{\raggedleft\let\newline\\\arraybackslash\hspace{0pt}}m{#1}}

\usepackage{multicol}
\usepackage[caption=false]{subfig}
\hypersetup{%
  colorlinks = true,
  linkcolor  = PineGreen!100!black
}

\usepackage{natbib}

\setbeamertemplate{itemize item}{\color{PineGreen}$\bullet$}
\setbeamertemplate{itemize subitem}{\color{PineGreen!25}$\circ$}
\setbeamercolor{enumerate item}{fg=PineGreen}
\setbeamercolor{caption name}{fg=PineGreen}

\setmainfont[]{Cardo} % sets the roman font
%\setsansfont[]{Alegreya} % sets the sans font % Vallkorn, Alegreya
\setsansfont[]{PT Sans} % sets the sans font
%\setsansfont[]{Segoe UI} % sets the sans font
\setmonofont[]{Consolas} % sets the monospace font

\setbeamertemplate{navigation symbols}{}
    \expandafter\def\expandafter\insertshorttitle\expandafter{%
      \insertshorttitle\hfill%
      \insertframenumber\,/\,\inserttotalframenumber}

\usetheme{Szeged}
\usecolortheme{spruce}

\title[Smart usage of context information for the analysis, design and generation of power-aware polices for mobile sensing apps]{Smart usage of context information for the analysis, design and generation of power-aware polices for mobile sensing apps}
\author[Rafael Pérez Torres]{Presented by: Rafael Pérez Torres\\[0.5cm] Thesis advisors:\\Dr. César Torres Huitzil\\Hiram Galeana Zapién, PhD}
\institute{LTI Cinvestav Tamaulipas}
%\date{\today}
\date{}

\begin{document}
\begin{frame}[plain]
  % {
  % \begin{center}
  % \includegraphics[scale=0.12]{../../../resources/images/vectors/cinvestav-logo-no-text}
  % \end{center}
  % }
  \begin{center}
  \includegraphics[scale=0.12]{../../../resources/images/vectors/cinvestav-logo-no-text}
  \end{center}
  \titlepage
  
\end{frame}

%\frame{\maketitle}

\begin{frame}{Table of contents}
	\tableofcontents[hideallsubsections]
\end{frame}

\section{Problem statement}
\subsection{Problem statement}

\begin{frame}[t]\frametitle{Introduction}
\begin{itemize}
  \item The popularity of mobile devices is a result of advances in their computation, \textbf{sensing}, and communication dimensions~\cite{Islam2014}.
  \begin{itemize}
    \item The sensing facilities improve interaction with user, turning them into \emph{omni-sensors} able to \emph{know} about its surrounding environment.
    \item Smartphones have become \emph{context-aware}, gaining understanding about user's activity and environment.
  \end{itemize}
  \item However, battery is not evolving at the same pace than the advances in other smartphone's characteristics~\cite{Kjaergaard2012}, growing only 5-10\% each year~\cite{Ma2012,Evarts2015}.
  \begin{itemize}
    \item The energy constraint becomes more critical when continuous access to sensors is needed, which is the core requirement of \textbf{mobile sensing applications}. 
  \end{itemize}
\end{itemize}
\end{frame}

\begin{frame}[t]\frametitle{Stages of mobile sensing applications}
\begin{figure}[tb]
  \centering
  \includegraphics[width=\textwidth]{../../../resources/images/vectors/msa-stages}
  \caption{Stages of mobile sensing applications}
  \label{fig:msa-stages}
\end{figure}
There is a tradeoff between the accuracy of context information retrieved and the associated energy consumption~\cite{Sim2014,Rachuri2012}.
How to face it?
\end{frame}

\begin{frame}\frametitle{Hypothesis}
\begin{tcolorbox}[title=Hypothesis,colframe=PineGreen]
Intelligent policies produced through context information built from sensors data can be employed to reduce the energy consumption in a mobile device when performing continuous sensor readings.
\end{tcolorbox}

{
\small
\begin{itemize}
  \item An intelligent policy is a special rule that defines how sensors should be selected and configured to reduce energy consumption and achieve the mobile sensing app's requirements.
  It is intelligent in terms of self-adaptness to changes detected in context information.
  \item This research work is aimed at employing GPS and inertial sensors data (accelerometer) for inferring context information in terms of mobility patterns.
  This context information will then be exploited to adapt sensors' operation and produce power savings.
\end{itemize}
}
\end{frame}

\begin{frame}\frametitle{Problem statement}
\begin{tcolorbox}[title=Problem statement: Mobility pattern identification,colframe=PineGreen]
\small
Given a set $V = \left\{v_{1}, v_{2}, \dotsc, v_{n}\right\}$ of data values read from sensor $S$ in the time interval $T  \in [t_{1}, t_{2}]$, identify the current mobility pattern $p_{S}$ that represents the activity of user.

\begin{equation}
  \text{PatternIdentifier}( V ) \longrightarrow{} p_{S} \in Patterns
\end{equation}

Where $Patterns$ is a set of patterns that represent an interesting state in user mobility, specifically the set $\left\{no\_movement, walking, running, vehicle\_transportation\right\}$.

\begin{figure}[tb]
  \centering
  \includegraphics[scale=0.5]{../../../resources/images/vectors/mobility-patterns-implications}
  \caption{Context information related to mobility patterns}
  \label{fig:mobility-patterns-implications}
\end{figure}

\end{tcolorbox}




\end{frame}

\begin{frame}\frametitle{Problem statement}
\begin{tcolorbox}[title=Problem statement: Policy generation,colframe=PineGreen]
\small
Given the set of detected mobility patterns $\mathcal{P} = \{ p_{S_1}, p_{S_2}, \ldots, p_{S_n} \}$ in data from sensors $\mathcal{S} = \{ S_1,S_2,\ldots, S_n \}$, accuracy required $a$, and physical constraints status $c$ of a mobile device, find a policy that select the proper set of sensors $\mathcal{S}_{new}$ and its associated configuration $\mathcal{S}_{new_{conf}}$  while meeting application requirements.

\begin{equation}
  \text{PolicyGeneration}( \mathcal{P}, a, c ) \longrightarrow{} \mathcal{S}_{new}, \mathcal{S}_{new_{conf}}
\end{equation}

The $\mathcal{S}_{new_{conf}}$ configuration is referred as the \emph{adaptive duty cycle} of associated sensor.
\end{tcolorbox}
\end{frame}

\begin{frame}\frametitle{Interaction between problems}
\begin{figure}[tb]
  \centering
  \includegraphics[width=\textwidth]{../../../resources/images/vectors/problems-incorporation}
  \caption{Interaction between the thesis work's problems}
  \label{fig:probems-incorporation}
\end{figure}
\end{frame}

\begin{frame}\frametitle{Objectives}
\begin{tcolorbox}[title=Main objective,colframe=PineGreen]
To reduce energy consumption in the mobile sensing apps, which perform continuous sensor readings, through self-adapting power-aware policies generated from context information obtained from sensors data.
\end{tcolorbox}

\begin{tcolorbox}[title=Particular objectives,colframe=PineGreen]
\small
\begin{itemize}
  \item To identify mobility patterns from context information obtained from an inertial sensor (accelerometer) and location providers (GPS, WPS).
  \item To generate policies for a self-adapting sensors' usage from identified mobility patterns, accuracy and energy requirements of mobile application, and status of mobile device's constraints. 
  \item To ease the development of mobile sensing applications that require user location tracking, i.e., LBS, isolating the complexity of sensors' access and the associated efficient energy management.
\end{itemize}
\end{tcolorbox}
\end{frame}


\begin{frame}[t]\frametitle{Problem's scenario}
\begin{figure}[tb]
  \centering
  \includegraphics[width=\textwidth]{../../../resources/images/vectors/scenario}
  \caption{Basic problem's scenario}
  \label{fig:scenario}
\end{figure}
\end{frame}


\section{Methodology}
\subsection{Methodology}
\begin{frame}\frametitle{Methodology}
\begin{enumerate}
  \item \textbf{Familiarization with state-of-art power-aware sensing related techniques}
  \item \textbf{Formal definition and selection of mobility patterns to be identified}
  \item \textbf{Research on pattern recognition algorithms focused on mobility patterns identification}
  \item \textbf{Design of the Pattern Identification Element (PIE)}
  \item Research on and proposition of adaptive policies for energy efficient usage of sensors
  \item Design of the Policy Generation Element (PGE)
  \item Development of a middleware involving the PIE and PGE for the Android platform
  \item Experimentation in terms of accuracy and energy efficiency
\end{enumerate}
\end{frame}


\section[State-of-art]{State of art related techniques}
\subsection{State of art}

\begin{frame}\frametitle{Taxonomy of state of art solutions}

\begin{figure}[tb]
  \centering
  \includegraphics[width=\textwidth]{../../../resources/images/vectors/approaches-taxonomy}
  \caption{Taxonomy of solutions, seen from the sensors adaptation's perspective}
  \label{fig:taxonomy}
\end{figure}
\end{frame}

\begin{frame}\frametitle{Distribution of approaches}
\begin{figure}[tb]
  \centering
  \includegraphics[scale=0.6]{../../../resources/images/vectors/approaches-distribution}
  \caption{Distribution of approaches across mobile platform's layers}
  \label{fig:distribution}
\end{figure}
\end{frame}


\begin{frame}[t]\frametitle{Characteristics of pure software approach solutions}
\begin{tcolorbox}[title=Distinctive characteristics of pure software solutions,colframe=PineGreen]
\small
\begin{itemize}
  \item \textbf{Optimization oriented (OO)}: Optimization orientation focused on minimizing energy consumption and/or the error in activity tracking.
  \item \textbf{Online learning (OL)}: Online learning from context information, enabling predictive features thanks to observance over long-time windows of sensory data.
  \item \textbf{User state oriented (US)}: Modeling of an enriched version of the context information, user state (US), for achieving full activity-awareness and ease the adaptation over the sensing dimension.
\end{itemize}
\end{tcolorbox}
\end{frame}


\begin{frame}[t]\frametitle{Framework for analyzing pure software solutions}
\begin{figure}[tb]
  \centering
  \includegraphics[width=\textwidth]{../../../resources/images/vectors/dual-taxonomy}
  \caption{Decomposition of solutions}
  \label{fig:dual-taxonomy}
  \end{figure}
\end{frame}


\section{Proposed solution}
\subsection{Proposed solution}
\begin{frame}[t]\frametitle{Proposed solution}
\begin{figure}[tb]
  \centering
  \includegraphics[width=\textwidth]{../../../resources/images/vectors/dual-taxonomy-ours}
  \caption{Decomposition of our solution}
  \label{fig:dual-taxonomy-ours}
  \end{figure}
\end{frame}

\begin{frame}[t]\frametitle{Proposed solution}
\begin{figure}[tb]
  \centering
  \includegraphics[width=0.9\textwidth]{../../../resources/images/vectors/solution-general-overview}
  \caption{Overview of current solution}
  \label{fig:solution}
\end{figure}
\end{frame}

\begin{frame}[t]\frametitle{The model of information learned in proposed solution}
\small
\begin{itemize}
  \item The overall problem can be divided into learning and detection of stay points, and the trajectory tracking.
\item The information can be represented through an expanded-spatial time model. 
\end{itemize}

\begin{figure}[tb]
  \centering
  \includegraphics[scale=0.45]{../../../resources/images/vectors/mobility-graph}
  \caption{Basic unit of information learned in the expanded spatial-time model (User state)}
  \label{fig:basic-unit-of-information-learned}
\end{figure}
\end{frame}

\begin{frame}[t]\frametitle{The model of information learned in proposed solution}
\begin{figure}[tb]
  \centering
  \includegraphics[width=\textwidth]{../../../resources/images/vectors/mobility-graph-larger-time-window-two-rows}
  \caption{The spatial-time models can be built and learned over longer time windows}
  \label{fig:information-learned-longer-time-windows}
\end{figure}
\end{frame}


\section{Important results}
\subsection{Important results}
\label{sub:important_results}
\begin{frame}[t]\frametitle{First steps towards expanded spatial-time model: a proof of concept}
\small
\begin{columns}
  \begin{column}{5cm}
    \begin{itemize}
    % \item A first step towards the acquisition of the expanded spatial-time model, locally at the mobile device, has been performed.
    \item A proof-of-concept experimentation for studying the feasibility of on-device learning the spatial-time model has been conducted.
    \item An \textbf{event-oriented} methodology was followed.
    %, which is a trend for \textbf{proactive} long-term applications focused on learning patterns from user activity.
    \item The methodology involved the adaptation of classic stay points algorithms~\cite{Montoliu2010,Zheng2011} for \emph{online} operation, while reducing memory footprint.
      % \item The learning of context information that can be employed for adapting sensory operations has been shown possible.  
  \end{itemize}
  \end{column}
  \begin{column}{7cm}
  \begin{figure}[tb]
  \scriptsize
  \centering
  \includegraphics[width=\columnwidth]{../../../resources/images/vectors/policy-creator-methodology}
  \caption{Logical workflow of the platform for on-device stay points detection}
  \label{fig:logical-workflow}
\end{figure}
  \end{column}
\end{columns}
\end{frame}


\begin{frame}[t]\frametitle{Results of early experimentation}
\small
\begin{itemize}
  \item The first experiment is aimed at ensuring the calculation of stay points locally at the smartphone~\footnote{Samsung Galaxy Note II, quadcore 1.6 GHz processor, 16 GB RAM, 3,100 mAh battery}.
  \item \emph{Spending power to save power} (process GPS data for learning mobility patterns and use this information for sensory adaptation) can be possible.
  %\item The parameters for running the stay point detection algorithms for all tests were set at 10 minutes for $\theta_{tmin}$, 1 hour for $\theta_{tmax}$ and 150 m for $\theta_{d}$.
\end{itemize}
\begin{table}[t]
\small{}
\centering
\resizebox{\textwidth}{!}{%
\begin{tabular}{@{}ccccccc@{}}
\toprule
\textbf{\begin{tabular}[c]{@{}c@{}}GPS reading\\ period\end{tabular}} & \textbf{Algorithm} & \textbf{\begin{tabular}[c]{@{}c@{}}Stay points\\ detected\end{tabular}} & \textbf{Average fixes} & \textbf{Maximum fixes} & \textbf{GPS accesses} & \textbf{Running time (hours)} \\ \midrule

1 minute &  Sigma Montoliou     & 27 & 70 & 652 & 2251 & 43.68 \\
1 minute &  Buffered Montoliou  & 26 & 79 & 685 & 2334 & 43.92 \\
\cmidrule{1-7}

3 minutes & Sigma Montoliou     & 47 & 27 & 248 & 1361 & 71.84 \\
3 minutes & Buffered Montoliou  & 28 & 41 & 248 & 1241 & 64.96 \\
\cmidrule{1-7}

5 minutes & Sigma Montoliou     & 43 & 25 & 154 & 1119 & 95.88 \\
5 minutes & Buffered Montoliou  & 45 & 21 & 154 & 1021 & 87.78 \\
\cmidrule{1-7}

Doubling 1-2-4-8-16 & Sigma Montoliou   & 59 & 17 & 80 & 1223 & 189.69 \\
Linear 1-3-5-9-12-15 &  Sigma Montoliou & 35 & 21 & 84 & 888 & 144.85 \\

\bottomrule
\end{tabular}
}
\caption{Results of the first experiment}
\label{tbl:experiment-1}
\end{table}
\end{frame}


% \section{Scientific publications produced}
\subsection{Publications}
\begin{frame}[t]\frametitle{Scientific products}
\begin{itemize}
  \item As a product of the analysis of related solutions, we have prepared a survey\footnote{\tiny \emph{Power management techniques in smartphone-based mobile sensing: a survey}, first review round, Pervasive and mobile computing (Elsevier)} covering:
  \begin{itemize}
    \item Smartphone-based sensing's characteristics.
    \item The power-awareness challenge in smartphone-based sensing.
    \item A taxonomy of the solutions for energy efficiency in smartphone-based sensing.
    \item The tendencies and open challenges of the field.
  \end{itemize}
  \item The aforementioned experimentation is work in progress for preparing an article focused on:
  \begin{itemize}
    \item On-device learning of mobility information with an event-oriented perspective.
    \item The usage of this information for a power efficient adaptation of the sensing dimension.
    \item Setting the base for further modules and strategies for boosting the learning of mobility patterns.
  \end{itemize}
\end{itemize}
\end{frame}


\section{Future work}
\subsection{Future work}
\begin{frame}[t]\frametitle{Future work}
    
  \definecolor{colorA}{gray}{0.85}
  \definecolor{colorB}{gray}{0.95}
  \definecolor{colorStep}{gray}{1}
  \definecolor{cyellow1}{RGB}{240,236,43}
  \definecolor{cgreen1}{RGB}{73,237,55}

  \newlength{\longitudCelda}
  \setlength{\longitudCelda}{75mm}

  \begin{table}[!h]
  {
    \scalebox{0.75}{
    \small
      \begin{tabular*}{16.5cm}{lp{\longitudCelda}*{7}{c}}
          & & \multicolumn{1}{ :c: }{\tiny 2014} & \multicolumn{3}{ c: }{\tiny 2015} & \multicolumn{3}{ c: }{\tiny 2016}  \\
          \cline{3-9}

          \multicolumn{2}{c:}{}
          & \multicolumn{1}{c:}{\tiny 3\textsuperscript{rd}} 
          & \tiny 1\textsuperscript{st} & { \tiny 2\textsuperscript{nd}} & \multicolumn{1}{c:}{ \tiny 3\textsuperscript{rd}}
          & \tiny 1\textsuperscript{st} & \tiny 2\textsuperscript{nd} & \multicolumn{1}{c:}{ \tiny 3\textsuperscript{rd}} \\
          \cline{3-9}

          % Step 1
          \rowcolor{colorStep}
          & \multicolumn{1}{c}{\scriptsize \textsc{Step I}}  & & & & & & \\
          
          \rowcolor{colorA}
          1 & \multicolumn{1}{p{\longitudCelda}:}{State-of-art reading} &
          \cellcolor{cgreen1} & \cellcolor{cgreen1} & &
          & & & \\

          \rowcolor{colorB}
          2 & \multicolumn{1}{p{\longitudCelda}:}{State-of-art works categorization} &
          & \cellcolor{cgreen1} & \cellcolor{cgreen1} &
          & & & \\

          \rowcolor{colorA}
          3 & \multicolumn{1}{p{\longitudCelda}:}{Documentation of information found (committee request)} &
          & & \cellcolor{cgreen1} &
          & & & \\

          % Step 2
          \rowcolor{colorStep}
          & \multicolumn{1}{c}{\scriptsize \textsc{Step II}}  & & & & & & & \\

          \rowcolor{colorB}
          4 & \multicolumn{1}{p{\longitudCelda}:}{Development of a mobile app for the Android platform that collects accelerometer and location data} &
          & & \cellcolor{cgreen1} & 
          \cellcolor{cgreen1} & & & \\

          \rowcolor{colorA}
          5 & \multicolumn{1}{p{\longitudCelda}:}{Analysis of data delivered by the mobile app} &
          & & & 
          \cellcolor{cyellow1} & & & \\

          \rowcolor{colorB}
          6 & \multicolumn{1}{p{\longitudCelda}:}{Creation of the formal definition of mobility pattern} &
          & & &
          \cellcolor{cyellow1} & & & \\

          \rowcolor{colorA}
          7 & \multicolumn{1}{p{\longitudCelda}:}{Selection of the mobility patterns to be recognizable by the system} &
          & & & 
          \cellcolor{cyellow1} & \cellcolor{cyellow1} & & \\

          % Step 3
          \rowcolor{colorStep}
          & \multicolumn{1}{c}{\scriptsize \textsc{Step III}}  & & & & & & & \\

          \rowcolor{colorB}
          8 & \multicolumn{1}{p{\longitudCelda}:}{Research on classification algorithms adequate to work with mobility patterns} &
          & & & 
          \cellcolor{cyellow1} & \cellcolor{cyellow1} & & \\

          \rowcolor{colorA}
          9 & \multicolumn{1}{p{\longitudCelda}:}{Definition of metrics for evaluating algorithms when executing on a mobile platform} &
          & & & 
          & \cellcolor[gray]{0.3} & & \\

          \rowcolor{colorB}
          10 & \multicolumn{1}{p{\longitudCelda}:}{Implementation of algorithms (including proper adaptations towards running on a mobile platform) for evaluation} &
          & & & 
          & \cellcolor[gray]{0.3} & \cellcolor[gray]{0.3} & \\

          \rowcolor{colorA}
          11 & \multicolumn{1}{p{\longitudCelda}:}{Selection of best algorithms according to metrics} &
          & & & 
          & & \cellcolor[gray]{0.3} & \\


          % Step 4
          \rowcolor{colorStep}
          & \multicolumn{1}{c}{\scriptsize \textsc{Step IV}}  & & & & & & & \\

          \rowcolor{colorB}
          12 & \multicolumn{1}{p{\longitudCelda}:}{Definition and modeling of parameters needed by the PIE} &
          & & & 
          \cellcolor{cyellow1} & \cellcolor[gray]{0.3} & \cellcolor[gray]{0.3} &
          \cellcolor[gray]{0.3}\\

          \rowcolor{colorA}
          13 & \multicolumn{1}{p{\longitudCelda}:}{Building of the PIE} &
          & & & 
          \cellcolor{cyellow1} & \cellcolor[gray]{0.3} & \cellcolor[gray]{0.3} &
          \cellcolor[gray]{0.3} \\

      \end{tabular*}
    }
  } % begin table
  \caption{Schedule of activities (each column represents a four months period)}
  \label{tbl-schedule}
  \end{table}
\end{frame}


\begin{frame}{Schedule}
  
  \newlength{\longCelda}
  \setlength{\longCelda}{75mm}
  \definecolor{colorA}{gray}{0.85}
  \definecolor{colorB}{gray}{0.95}
  \definecolor{colorStep}{gray}{1}
  \definecolor{cyellow1}{RGB}{240,236,43}

  \begin{table}[!h]
  {
    \scalebox{0.8}{
    \small
      \begin{tabular*}{16.5cm}{lp{\longCelda}*{6}{c}}
          & & \multicolumn{3}{ :c: }{\tiny 2016} & \multicolumn{3}{ c: }{\tiny 2017} \\
          \cline{3-8}

          \multicolumn{2}{ c: }{}
          & \multicolumn{1}{c}{\tiny 1\textsuperscript{st}} & { \tiny 2\textsuperscript{nd}} & \multicolumn{1}{c:}{ \tiny 3\textsuperscript{rd}}
          & \tiny 1\textsuperscript{st} & \tiny 2\textsuperscript{nd} & \multicolumn{1}{c:}{ \tiny 3\textsuperscript{rd}} \\
          \cline{3-8}

          % Step 5
          \rowcolor{colorStep}
          & \multicolumn{1}{c}{\scriptsize \textsc{Step V}}  & & & & & & \\
          
          \rowcolor{colorA}
          14 & \multicolumn{1}{p{\longCelda}:}{Creation of the formal definition of policy} &
          & & \cellcolor[gray]{0.3} &
          & & \\

          \rowcolor{colorB}
          15 & \multicolumn{1}{p{\longCelda}:}{Research and evaluation of techniques for generation and adaption of policies} &
          & & \cellcolor[gray]{0.3} & 
          \cellcolor[gray]{0.3} & & \\

          \rowcolor{colorA}
          16 & \multicolumn{1}{p{\longCelda}:}{Design and execution of experiments with techniques applied to use cases} &
          & & \cellcolor[gray]{0.3} & 
          \cellcolor[gray]{0.3} & & \\

          \rowcolor{colorB}
          17 & \multicolumn{1}{p{\longCelda}:}{Selection of best techniques} &
          & & &
          \cellcolor[gray]{0.3} & & \\


          % Step 6
          \rowcolor{colorStep}
          & \multicolumn{1}{c}{\scriptsize \textsc{Step VI}}  & & & & & & \\
          
          \rowcolor{colorA}
          18 & \multicolumn{1}{p{\longCelda}:}{Definition and modeling of parameters needed by the PGE} &
          & & &
          & \cellcolor{cyellow1} & \\

          \rowcolor{colorB}
          19 & \multicolumn{1}{p{\longCelda}:}{Building of the PGE} &
          & & &
          & \cellcolor{cyellow1} & \\



      % Step 7
          \rowcolor{colorStep}
          & \multicolumn{1}{c}{\scriptsize \textsc{Step VII}}  & & & & & & \\
          
          \rowcolor{colorA}
          20 & \multicolumn{1}{p{\longCelda}:}{Analysis of elements involved into software abstractions} &
          & & &
          & \cellcolor[gray]{0.3} & \\

          \rowcolor{colorB}
          21 & \multicolumn{1}{p{\longCelda}:}{Research on Android API for specialized components} &
          & & &
          & \cellcolor[gray]{0.3} & \\

          \rowcolor{colorA}
          22 & \multicolumn{1}{p{\longCelda}:}{Development of the middleware} &
          & & &
          & \cellcolor[gray]{0.3} & \\


      \end{tabular*}
     }
  } % begin table
  \caption{Schedule of activities (each column represents a four months period)}
  \label{tbl-schedule-2}
  \end{table}
\end{frame}

\begin{frame}{Schedule}
  
  \newlength{\longCell}
  \setlength{\longCell}{75mm}
  \definecolor{colorA}{gray}{0.85}
  \definecolor{colorB}{gray}{0.95}
  \definecolor{colorStep}{gray}{1}

  \begin{table}[!h]
  {
    \scalebox{0.8}{
    \small
      \begin{tabular*}{16.5cm}{lp{\longCell}*{5}{c}}
          & & \multicolumn{3}{ :c: }{\tiny 2017} & \multicolumn{2}{ c: }{\tiny 2018} \\
          \cline{3-7}

          \multicolumn{2}{ c: }{}
          & \multicolumn{1}{c}{\tiny 1\textsuperscript{st}} & { \tiny 2\textsuperscript{nd}} & \multicolumn{1}{c:}{ \tiny 3\textsuperscript{rd}}
          & \tiny 1\textsuperscript{st} & \tiny 2\textsuperscript{nd} \\
          \cline{3-7}

          % Step 8
          \rowcolor{colorStep}
          & \multicolumn{1}{c}{\scriptsize \textsc{Step VIII}}  & & & \\
          
          \rowcolor{colorA}
          23 & \multicolumn{1}{p{\longCell}:}{Definition of experiments targeting accuracy and energy consumption axes} &
          & & \cellcolor[gray]{0.3} &
          & \\

          \rowcolor{colorB}
          24 & \multicolumn{1}{p{\longCell}:}{Development of needed sample mobile apps for running experimentation} &
          & & \cellcolor[gray]{0.3} &
          & \\

          \rowcolor{colorA}
          25 & \multicolumn{1}{p{\longCell}:}{Execution of experimentation} &
          & & \cellcolor[gray]{0.3} &
          \cellcolor[gray]{0.3} & \\

          \rowcolor{colorB}
          26 & \multicolumn{1}{p{\longCell}:}{Results analysis} &
          & & &
          \cellcolor[gray]{0.3} & \\

      \end{tabular*}
     }
  } % begin table
  \caption{Schedule of activities (each column represents a four months period)}
  \label{tbl-schedule-3}
  \end{table}
\end{frame}


\begin{frame}{Schedule}
  
  \newlength{\longCella}
  \setlength{\longCella}{50mm}
  \definecolor{colorA}{gray}{0.85}
  \definecolor{colorB}{gray}{0.95}
  \definecolor{colorStep}{gray}{1}

  \definecolor{cyellow1}{RGB}{240,236,43}
  \definecolor{cgreen1}{RGB}{73,237,55}


  \begin{table}[!h]
  {
    \scalebox{0.75}{
    %\scalebox{0.6}{
    \small
      \begin{tabular*}{16.5cm}{lp{\longCella}*{12}{c}}
          & & \multicolumn{1}{ :c: }{\tiny 2014} & \multicolumn{3}{ c: }{\tiny 2015} & \multicolumn{3}{ c: }{\tiny 2016}  & \multicolumn{3}{ c: }{\tiny 2017} & \multicolumn{2}{ c }{\tiny 2018} \\
          \cline{3-14}

          \multicolumn{2}{c:}{}
          & \multicolumn{1}{c:}{\tiny 3\textsuperscript{rd}} 
          & \tiny 1\textsuperscript{st} & { \tiny 2\textsuperscript{nd}} & \multicolumn{1}{c:}{ \tiny 3\textsuperscript{rd}}
          & \tiny 1\textsuperscript{st} & \tiny 2\textsuperscript{nd} & \multicolumn{1}{c:}{ \tiny 3\textsuperscript{rd}}
          & \tiny 1\textsuperscript{st} & \tiny 2\textsuperscript{nd} & \multicolumn{1}{c:}{ \tiny 3\textsuperscript{rd}}
          & \tiny 1\textsuperscript{st} & \tiny 2\textsuperscript{nd} \\
          \cline{3-14}

          % Step 1
          \rowcolor{colorStep}
          & \multicolumn{1}{c}{\scriptsize \textsc{Required tasks}}  & & & & & & & & & & & & \\
          
          \rowcolor{colorA}
          A & \multicolumn{1}{p{\longCella}:}{Related subject courses} &
          \cellcolor{cgreen1} & \cellcolor{cgreen1} & \cellcolor{cgreen1} &
          & & &
          & & &
          & & \\

          \rowcolor{colorB}
          B & \multicolumn{1}{p{\longCella}:}{Research articles submission} &
          & & & \cellcolor{cyellow1} *
          & & \cellcolor[gray]{0.3} &
          & & \cellcolor[gray]{0.3} &
          & & \\

          \rowcolor{colorA}
          C & \multicolumn{1}{p{\longCella}:}{Thesis writing} &
          \cellcolor{cgreen1} & & &
          \cellcolor{cyellow1} & & &
          \cellcolor[gray]{0.3} & & &
          \cellcolor[gray]{0.3} & \cellcolor[gray]{0.3} & \cellcolor[gray]{0.3} \\

      \end{tabular*}
    }
  } % begin table
  \caption{Schedule of required activities}
  \label{tbl-schedule-4}
  \end{table}

  * Survey was out of original schedule, however, it brings added value to our work and also acts as a scientific booster for improving and achieving our solution.
\end{frame}


\section{Conclusions}
\subsection{Conclusions}
\begin{frame}\frametitle{Conclusions}

\begin{itemize}
  \item A brief analysis of the problem addressed by this thesis work has been presented.
  \item A summary of related state of art solutions, as suggested by committee, has been covered:
  \begin{itemize}
    \item A taxonomy for categorizing of the solutions.
    \item A new framework for decomposing and studying them. 
  \end{itemize}
  \item The general approach of our solution, which is based on an expanded spatial-time model for identifying and learning user mobility, has been introduced.
  \item An experimental step that shows the feasibility of performing the on-device building of such expanded models has been also presented. 
\end{itemize}
\end{frame}

\begin{frame}\frametitle{}
\begin{center}
{
\Huge
Thank You\\ for your attention!
}
\end{center}

\end{frame}



\subsection{References}
\bibliographystyle{plain}

\bibliography{../../../resources/references/bibliography}

\begin{frame}[t]\frametitle{Montoliou's algorithm for stay points detection~\cite{Montoliu2010}}
\begin{center}
\scalebox{0.85}{
\begin{minipage}{\linewidth}


\scriptsize{}
\begin{algorithmic}[1]
\REQUIRE A GPS trajectory $T=\{p_{1},p_{2},\ldots,p_{N}\}$, a distance threshold $\theta_{d}$, a minimum time threshold $\theta_{tmin}$, and a maximum time threshold $\theta_{tmax}$.
\ENSURE A set of points of interest $\Pi$
\STATE $i \leftarrow 1$ 
\STATE $\Pi \leftarrow \emptyset$
\WHILE {$i<N$}
  \STATE $j=i+1$
  \WHILE {$j<N$}
    \STATE $t\leftarrow \text{\textbf{timeDifference}}(p_{j},p_{j-1})$
    \IF{ $t>\theta_{tmax}$}
      \STATE $i \leftarrow j$
      \STATE \textbf{break}
    \ENDIF
    \STATE $d\leftarrow distance(p_{i},p_{j})$
    \IF{ $d>\theta_{d}$}
      \STATE $t\leftarrow \text{\textbf{timeDifference}}(p_{i},p_{j-1})$
      \IF{ $t>\theta_{tmin}$}
        \STATE $\pi.\text{lat} = \sum_{k=i}^{j-1} \frac{p_{k}.\text{lat}}{|j-1-i|}$
        \STATE $\pi.\text{lon} = \sum_{k=i}^{j-1} \frac{p_{k}.\text{lon}}{|j-1-i|}$
        \STATE $\pi.\text{at}  = p_{i}.\text{ts}$ 
        \STATE $\pi.\text{dt}  = p_{j-1}.\text{ts}$
        \STATE $\Pi \leftarrow \Pi \cup \pi$
      \ENDIF
      \STATE $i\leftarrow j$
      \STATE \textbf{break}
    \ENDIF
    \STATE $j\leftarrow j+1;$
  \ENDWHILE
\ENDWHILE
\end{algorithmic}
% \caption{Montoliou's algorithm for stay points detection~\cite{Montoliu2010} \label{alg:montoliou_algorithm}}

\end{minipage}%
}
\end{center}
\end{frame}


\begin{frame}[t]\frametitle{State-of-art solutions}
\begin{table}
    \centering
    \scriptsize{}
    \resizebox{\textwidth}{!} { 
    \begin{tabular}{C{0.11\textwidth}C{0.2\textwidth}C{0.17\textwidth}C{0.15\textwidth}C{0.09\textwidth}C{0.05\textwidth}C{0.05\textwidth}C{0.05\textwidth}}
    \toprule
    \textbf{Name} & \textbf{Variants} & \textbf{Machine learning technique} & \textbf{Sensors involved} & \textbf{Complexity} & \textbf{OL} & \textbf{OO} & \textbf{US} \\
    \midrule

    
    % Category Basic context policy (short time window)
    \emph{G-Sense}      \cite{Perez2010} & User behavior learning (DC) & SDR & GPS & low \\

    \cmidrule(l){1-8}
    Perez-Torres           \cite{Perez-Torres2012} & User behavior learning (DC) & SDR & GPS & low \\

    \cmidrule(l){1-8}
    \emph{SenseLess}    \cite{Abdesslem2009} & User behavior learning (DC), Sensor replacement (CR, DR) & SDR & WPS, GPS, ACC & low \\
    
    \cmidrule(l){1-8}
    \emph{SensTrack}    \cite{Zhang2013} & User behavior learning (DC), Sensor replacement (CR, DR) & SDR & ACC, orientation sensor, GPS, WPS & low \\
    
    \cmidrule(l){1-8}
    Man and Ngai           \cite{Man2014} & User behavior learning (DC, VI), Sensor replacement (CR) & SDR & ACC, magnetic field sensor, GPS & low \\

    \cmidrule(l){1-8}
    \emph{EnLoc}        \cite{Constandache2009} & User behavior learning (DC, VI), Sensor replacement (DR) & SDR, Mobility Tree & WPS, GPS, cellular ID & medium & & \checkmark \\
    
    \cmidrule(l){1-8}
    \emph{EnTracked}    \cite{Kjaergaard2009} & User behavior learning (DC), Sensor replacement (CR) & SDR & ACC, GPS & medium & & \checkmark \\
    \bottomrule
    \end{tabular}
    }
    \protect\caption{Pure software solutions. (OL: Online Learning from user data, OO: Optimization Oriented solution, US: User State context insight)\label{tab:works-software-approach-1}}
\end{table}
\end{frame}

\begin{frame}[t]\frametitle{State-of-art solutions}
\begin{table}
    \centering
    \scriptsize{}
    \resizebox{\textwidth}{!} { 
    \begin{tabular}{C{0.11\textwidth}C{0.2\textwidth}C{0.17\textwidth}C{0.15\textwidth}C{0.09\textwidth}C{0.05\textwidth}C{0.05\textwidth}C{0.05\textwidth}}
    \toprule
    \textbf{Name} & \textbf{Variants} & \textbf{Machine learning technique} & \textbf{Sensors involved} & \textbf{Complexity} & \textbf{OL} & \textbf{OO} & \textbf{US} \\
    \midrule

    % Category Medium context-aware policy
    Alvarez, Morillo           \cite{AlvarezDeLaConcepcion2014,Morillo2015} & --- & Ameva algorithm & ACC & medium \\
    
    \cmidrule(l){1-8}
    Mazilu           \cite{Mazilu2013} & Sensor replacement (CR) & DT & Temperature, humidity, pressure & medium \\
    % Not given           \cite{Mazilu2013} & Sensor replacement (CR) & DT (C4.5) & Temperature, humidity, pressure & medium \\

    \cmidrule(l){1-8}
    Srinivasan           \cite{Srinivasan2012} & User behavior learning (DC) & DT & ACC & medium \\
    
    \cmidrule(l){1-8}
    Khalifa           \cite{Khalifa2015} & Sensor replacement (CR) & KNN & Model of ACC-based harvesting device & medium \\
    \bottomrule
    \end{tabular}
    }
    \protect\caption{Pure software solutions. (OL: Online Learning from user data, OO: Optimization Oriented solution, US: User State context insight)\label{tab:works-software-approach-2}}
\end{table}
\end{frame}


\begin{frame}[t]\frametitle{State-of-art solutions}
\begin{table}
    \centering
    \scriptsize{}
    \resizebox{\textwidth}{!} { 
    \begin{tabular}{C{0.11\textwidth}C{0.2\textwidth}C{0.17\textwidth}C{0.15\textwidth}C{0.09\textwidth}C{0.05\textwidth}C{0.05\textwidth}C{0.05\textwidth}}
    \toprule
    \textbf{Name} & \textbf{Variants} & \textbf{Machine learning technique} & \textbf{Sensors involved} & \textbf{Complexity} & \textbf{OL} & \textbf{OO} & \textbf{US} \\
    \midrule

    % Category Long time window
    \emph{SensLoc}         \cite{Kim2010} & User behavior learning (DC, RD), Sensor replacement (CR) & SDR & Wi-Fi fingerprinting, GPS, ACC & medium & \checkmark & \\
    
    \cmidrule(l){1-8}
    \emph{CAPS}         \cite{Paek2011} & User behavior learning (DC, RD), Sensor replacement (CR) & SDR & GPS, cellular ID & medium & \checkmark \\
    % SDR (Smith-Waterman algorithm) CR because location is processed alternatively with this algorithm

    \cmidrule(l){1-8}
    \emph{RAPS}         \cite{Paek2010} & User behavior learning (DC, RD), Sensor replacement (CR, DR) & SDR & WPS, GPS, ACC, Bluetooth, cellular ID & medium & \checkmark \\
    
    % In-between Category Optimization + long time window
    \cmidrule(l){1-8}
    \emph{A-Loc}        \cite{Lin2010} & User behavior learning (DC, RD), Sensor replacement (CR, DR) & HMM, Bayesian estimation framework & GPS, WPS, Bluetooth, cellular ID & medium & \checkmark & \checkmark  \\
    
    \cmidrule(l){1-8}
    \emph{SmartDC}      \cite{Chon2011} & User behavior learning (DC, RD), Sensor replacement (CR, DR) & HMM and LZ predictor & GPS, WPS, Wi-Fi and cellular ID fingerprinting & medium & \checkmark & \checkmark \\
    
    \bottomrule
    \end{tabular}
    }
    \protect\caption{Pure software solutions. (OL: Online Learning from user data, OO: Optimization Oriented solution, US: User State context insight)\label{tab:works-software-approach-3}}
\end{table}
\end{frame}

\begin{frame}[t]\frametitle{State-of-art solutions}
\begin{table}
    \centering
    \scriptsize{}
    \resizebox{\textwidth}{!} { 
    \begin{tabular}{C{0.11\textwidth}C{0.2\textwidth}C{0.17\textwidth}C{0.15\textwidth}C{0.09\textwidth}C{0.05\textwidth}C{0.05\textwidth}C{0.05\textwidth}}
    \toprule
    \textbf{Name} & \textbf{Variants} & \textbf{Machine learning technique} & \textbf{Sensors involved} & \textbf{Complexity} & \textbf{OL} & \textbf{OO} & \textbf{US} \\
    \midrule

    % Category Full context-aware policy
    \emph{Jigsaw}       \cite{Lu2010} & User behavior learning (DC), Sensor replacement (CR) & Microphone: NB with Gaussian Mixture Model (GMM). ACC: DT. GPS: MDP. & ACC, Microphone, GPS & high & & \checkmark & \checkmark \\
    
    \cmidrule(l){1-8}
    Donohoo           \cite{Donohoo2014} & User behavior learning (DC) & Several. KNN and NN selected as best. & ACC, GPS, WPS, cellular ID, light, device data, mobile app requirements & high & & & \checkmark \\

    \cmidrule(l){1-8}
    \emph{EEMSS}        \cite{Wang2009} & User behavior learning (DC), Sensor replacement (CR, DR) & GPS and ACC: SDR. \newline Microphone: SSCH algorithm. & ACC, microphone, GPS & high & & & \checkmark \\

    \cmidrule(l){1-8}
    \emph{iLoc}         \cite{MaY2009} & User behavior learning (RD), Sensor replacement (CR) & HMM & Wi-Fi \& GSM fingerprinting & high & \checkmark & & \checkmark  \\
    
    \cmidrule(l){1-8}
    Yurur           \cite{Yurur2014} & User behavior learning (DC, RD) & HMM & ACC & high & \checkmark & & \checkmark \\
    
    \cmidrule(l){1-8}
    \emph{FreeTrack}    \cite{Chon2014} & User behavior learning (DC, RD), Sensor replacement (CR, DR) & HMM & GPS, Wi-Fi, cellular ID, battery status & high & \checkmark & & \checkmark \\
    
    \bottomrule
    \end{tabular}
    }
    \protect\caption{Pure software solutions. (OL: Online Learning from user data, OO: Optimization Oriented solution, US: User State context insight)\label{tab:works-software-approach-4}}
\end{table}
\end{frame}

\end{document}
