\documentclass{article}
\usepackage{mathpazo}
\usepackage[T1]{fontenc}
\usepackage[utf8]{inputenc}
\usepackage{eurosym}
%\usepackage{natbib}

\usepackage[table]{xcolor}

\usepackage{booktabs}
\usepackage{array}
\newcolumntype{L}[1]{>{\raggedright\let\newline\\\arraybackslash\hspace{0pt}}m{#1}}
\newcolumntype{C}[1]{>{\centering\let\newline\\\arraybackslash\hspace{0pt}}m{#1}}
\newcolumntype{R}[1]{>{\raggedleft\let\newline\\\arraybackslash\hspace{0pt}}m{#1}}

\date{Doctorate Seminar}
\begin{document}

%\makeatletter

\title{H2020 Research and Innovation Programme}
\author{Some guys here}

\maketitle
\abstract{This document presents the objectives, requirements, scope, characteristics, range of funding and expected results for proposals attending the Horizon 2020 research and innovation programme.
}


\section{Self definition}
Horizon 2020 is the biggest EU Research and Innovation programme ever with nearly \EUR{80} billion of funding available over 7 years (2014 to 2020) – in addition to the private investment that this money will attract. It promises more breakthroughs, discoveries and world-firsts by taking great ideas from the lab to the market.
[28 members, 80 000 000 000 000 then 2 857 142 857 142.857, 2 billion per country]

\section{Objectives}
H2020 also supports SME
H2020 supports SMEs with a new instrument that runs throughout various funded research and innovation fields, enhances EU international research and Third Country participation, attaches high importance to integrate social sciences and humanities encourages to develop a gender dimension in project.

Horizon 2020 is divided into 3 three pillars and 2 specific objectives corresponding to its main priorities:

\begin{itemize}
    \item Excellent Science.
    \item Industrial Leadership.
    \item Societal Challenges.
    \item Specific objective \emph{Spreading excellence \& widening participation}.
    \item Specific objective \emph{Science with and for society}.
\end{itemize}

\section{Requirements}
It requires a \textbf{full} proposal...
The proposal itself consists of 2 main parts: administrative forms (structured information of the basic administrative data, declarations of partners, organisations and contact persons, etc.) and the technical annex, which is the detailed description of the planned research and innovation project outlining work packages, costs, etc.
Further mandatory or optional annexes (e.g. supporting documents for ethics issues) can be required by the call and the given topic, as shown in the submission system.

Proposals can be submitted multiple times before the call deadline specified in the information package of the call, available from the Participant Portal. Calls deadlines are fixed and will normally not be extended . Only the most recently submitted version will be evaluated, where as each newly submitted version overwrites the previous one. After the call deadline, the proposal can no longer be modified and no further participants can be invited. Practise the proposal submission procedure well before the deadline to ensure a risk free submission of your proposal and proper correction of errors and warnings. After the deadline, the proposal remains available in read only mode and can be accessed by the coordinator and the proposal participants before the deadline

\section{Scope}
Areas of participation:
\begin{itemize}
	\item Agriculture \& Forestry
	\item Aquatic Resources
	\item Bio-based Industries
    \item Biotechnology
    \item Energy
    \item Environment \& Climate Action
    \item Food \& Healthy Diet
    \item Funding Researchers
    \item Health
    \item ICT Research \& Innovation
    \item Innovation
    \item International Cooperation
    \item Key Enabling Technologies
    \item Partnerships with Industry and Member States
    \item Raw Materials
    \item Research Infrastructures
    \item Security
    \item SMEs
    \item Social Sciences \& Humanities
    \item Society
    \item Space
    \item Transport
\end{itemize}

\section{Characteristics (The Workflow of a proposal)}
The Participant Portal is your entry point for electronic administration of EU-funded research and innovation projects, and hosts the services for managing your proposals and projects throughout their lifecycle.

\subsection{The application process}

\subsubsection{Submit your proposal}
If you wish to respond to a call, you must submit a proposal before the deadline. The Participant Portal has clear instructions to guide you through the process. The system is simpler than ever – no more paper! All proposals are submitted online.

\subsubsection{Find your partners}
Many calls require a team of at least three partners. If you need help to identify a potential partner with particular competences, facilities or experience, use the partner search options.

\subsubsection{Evaluation by experts}
Once the deadline has passed, all proposals are evaluated by a panel of independent specialists in their fields. The panel checks each proposal against a list of criteria to see if it should receive funding.

For evaluating a proposal, it first must be admissible and elegible:
\begin{itemize}
	\item Admissible:
	\begin{itemize}
		\item Is submitted via the official online submission system before the call deadline.
		\item Is complete – accompanied by the relevant administrative forms, proposal description and any supporting documents specified in the call.
		\item Is readable, accessible and printable.
		\item Grant proposals must include a draft plan for the exploitation and dissemination of the results, unless otherwise specified in the call conditions. The draft plan is not required for proposals at the first stage of two-stage procedures.
		\item Furthermore, page limits will apply to proposals/applications. Your proposal must not exceed the maximum number of pages indicated in the proposal template (70 pages). 
	\end{itemize}
	
	\item Elegible:
	\begin{itemize}
		\item Its contents are in line with the topic description in the call.
		\item it involves enough of the right participants and meets Standard eligibility criteria and any other eligibility conditions set out in the call or topic page
	\end{itemize}
\end{itemize}

In the outcome of evaluation for a given proposal, the experts score each award criterion on a scale from 0 to 5:
\begin{itemize}
	\item 0 Proposal does not meet the criterion at all or cannot be assessed due to missing or incomplete information
    \item 1 Poor – serious weaknesses
    \item 2 Fair – goes some way to meeting criterion, but with significant weaknesses
    \item 3 Good – but with a number of shortcomings
    \item 4 Very good – but with a small number of shortcomings
    \item 5 Excellent – meets criterion in every relevant respect. Any shortcomings are minor
\end{itemize}

\subsubsection{Grant agreement}
Once a proposal passes the evaluation stage (\textbf{five months’ duration}), applicants are informed about the outcome. The European Commission then draws up a grant agreement with each participant. The grant agreement confirms what research \& innovation activities will be undertaken, the project duration, budget, rates and costs, the European Commission's contribution, all rights and obligations and more. The time limit for signing the grant agreements is generally \textbf{three months}.


\section{Range of funding}
\$ for project

\subsection{SME}
\label{sub:sme}
For SME\footnote{SME, Small Medium-sized Enterprises}
\subsubsection{Feasibility assessment (phase 1) - optional}

Funding is available for: exploring and assessing the technical feasibility and commercial potential of a breakthrough innovation that a company wants to exploit and commercialize.

Activities funded could be: risk assessment, design or market studies, intellectual property exploration; the ultimate goal is to put a new product, service or process in the market, possibly through an innovative application of existing technologies, methodologies, or business processes.

The project should be aligned to the business strategy, helping internal growth or targeting a transnational business opportunity.

Amount of funding: lump sum of \EUR{50,000} (per project, not per participating business).

Duration: typically around 6 months

Outcome: The outcome of a phase 1 project is a feasibility study (technical and commercial), including a business plan.

Should the conclusion of the study be that the innovative concept has the potential to be developed to the level of investment readiness/market maturity, but requires additional funding in view of commercialisation, the SME can apply for Phase 2 support.


\subsubsection{Innovation project (phase 2)}

Funding is available for: innovation projects underpinned by a sound and strategic business plan (potentially elaborated and partially funded through phase 1 of the SME Instrument).

Activities funded in phase 2 can be of several types: prototyping, miniaturisation, scaling-up, design, performance verification, testing, demonstration, development of pilot lines, validation for market replication, including other activities aimed at bringing innovation to investment readiness and maturity for market take-up.

Amount of funding: in the indicative range of \EUR{500,000} – \EUR{2.5} million or more (covering up to 70\% of eligible costs, or in exceptional, specific cases up to 100\%).

Duration: typically around 1 to 2 years

Outcomes: a new product, process or service that is ready to face market competition; a business innovation plan incorporating a detailed commercialisation strategy and a financing plan in view of market launch (e.g. on how to attract private investors, if applicable).


\subsubsection{Commercialisation (phase 3)}
With the view of facilitating the commercial exploitation of the innovation activities resulting from phase 1 or phase 2, specific activities will be proposed. These can include support for further developing investment readiness, linking with private investors and customers through brokerage activities, assistance in applying for further EU risk finance, and a range of other innovation support activities and services offered via the Enterprise Europe Network (EEN).


\subsubsection{Coaching}
Innovation and Business development coaching is proposed in parallel throughout phases 1 and 2 to help SMEs:
\begin{itemize}
	\item Enhance the company's innovation capacity
    \item Align the project to the identified business development strategy
    \item Develop the commercial/economic impact and long term sustainability.
\end{itemize}

Coaching will be provided by experienced business coaches, selected through the Entreprise Europe Network (EEN).

\section{Expected results}
Papers, patents?

The papers produced should be open access

The coordinator must submit the deliverables identified in Annex 1, in accordance with the timing and condi tions set out in that Annex. ‘Deliverables' are additional outputs (e.g. information, special report, a technical diagram brochure, list, a software milestone or other building block of the action ) that must be produced at a given moment during the action (normally not at the same time as the periodic/final reports). ( ‘Milestones' are, by contrast, control points in the action that help to chart progress. They may correspond to the completion of a key deliverable, allowing the next phase of the work to beg in or be needed at intermediary points)

Unless it goes against their legitimate interests, each beneficiary must — as soo	n as possible — ‘ disseminate ' its results by disclosing them to the public by appropriate means (other than those resulting from protecting or exploiting the results), including in scientific publications (in any medium)
\end{document}