\abstract{
% The popularity of modern smartphones and mobile sensing apps is a consequence of the mobility and increasing storage, computing and communication features of these devices. 
% Because of this, the smartphone has become a full time companion of user. 
% Nevertheless, the energy consumption in mobile sensing apps represents an open problem in the research that has been addressed from several perspectives since its relevance through all layers of the mobile platform.

% This technical report describes the initial stage of the work thesis focused in solving the energy issue from a software perspective.
% The report describes the advances developed during the first period of the doctoral program, including a description of the problem, and focused in the review of the state of art which presents a preliminary taxonomy of works addressing this problem.
% Also, the review of the method pursued in this research for solving the energy consumption problem is also presented.

This technical report presents the fundamental aspects and advances developed in the work thesis titled \emph{Smart Usage of Context Information for the Analysis, Design, and Generation of Power-Aware Policies for Mobile Sensing Apps} during the first period of the doctoral program.
Most of these advances refer to the state of art research, and include the description of a preliminary taxonomy of works aiming to solve the energy consumption problem by the mobile sensing apps.

Also, the review of the method pursued in this research for solving the energy consumption problem is presented.
}