\section{Conclusions}
\label{sec:conclusions}
% Mobile devices, like the smartphones, have become sophisticated computing elements that have the capability of sensing the environment. 
% Numerous examples of mobile apps leveraging the sensing and computing features of the smartphones can be found in the market, being known as mobile sensing apps.

% The mobile sensing apps expand the benefits that smartphones can offer, but also have a great impact in the energy consumption of the devices.
% The energy consumption represent a complex and cross layer issue that has been tried to solve from several perspectives.

This technical report has shown the advances performed in the thesis work during the first period of the doctoral program.

In particular, the report has described a preliminary taxonomy of works found in literature aiming to solve the energy consumption problem in the mobile sensing apps.
This taxonomy classifies the works depending on the layer of the mobile platform that they aim to modify, producing three big families of techniques, namely the pure software, the hardware-software, and the pure hardware approaches.


Finally, the document has shown a review of a pure software solution proposed in the research for addressing the energy problem.
This solution aims to increase the smartness level of the mobile platform, becoming aware of the environment for making informed decisions about the sensors usage, which will result in energy savings and in the ease of mobile sensing apps development.
