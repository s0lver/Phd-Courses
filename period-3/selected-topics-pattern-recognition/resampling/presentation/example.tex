\documentclass[aspectratio=169]{beamer}
\usepackage[utf8]{inputenc}
\usepackage[T1]{fontenc}
\usepackage{lipsum, lmodern}
\usetheme{Metro} % or Median, or Metro, or PraterStreet, or Milano

\author{Rafael Pérez-Torres}
\title{Re-sampling and cross-validation}
\institute{Cinvestav Tamaulipas}
\date{\today}

\begin{document}
\frame{\maketitle}
\begin{frame}{Table of contents}
	\tableofcontents
\end{frame}

\section{Introduction}
\begin{frame}{Introduction}

{\Large{} Why cross validation and resampling?} 

Cross-validation and resampling methods are validation techniques helpful for:
\begin{itemize}
	\item Selecting model.
	\begin{itemize}
		\item Almost all pattern recognition techniques needs one or more parameters.
		\item How to select the \emph{optimal} parameters?
	\end{itemize}
	\item Classifiers performance evaluation.
	\begin{itemize}
		\item Once the model is selected, how to estimate its performance?
		\item The goal is the real error rate, but this is only achievable by performing classification over the whole population.
	\end{itemize}
\end{itemize}
\end{frame}

\begin{frame}{Introduction}

{\Large{} Why cross-validation and resampling?} 

\begin{itemize}
	\item Usually the available dataset size is not as large as we would want.
	\item One approach would be selecting the entire dataset for the classifier training and evaluation, but:
	\begin{itemize}
		\item This would overfit training data.
		\item The error rate estimate might be really optimistic.
	\end{itemize}
\end{itemize}

{\Large{} Resampling and cross-validation techniques to the rescue!} 
\end{frame}

\section{Cross-validation techniques}
\begin{frame}{Cross-validation techniques}

% {\Large{} Hold out cross validation} 
\begin{itemize}
	\item Basically, they divide available dataset on two subsets, one for training, the other for testing the classifier.
	\item Subsets are mutually exclusive, the instance $x_i$ can be only in one of these subsets.
\end{itemize}
\end{frame}


\section{Re-sampling techniques}
\begin{frame}{Re-sampling techniques}

% {\Large{} Hold out cross validation} 
\begin{itemize}
	\item Features here!
\end{itemize}
\end{frame}



\begin{frame}{Enumerate}
\begin{enumerate}
\item Here you can see an enumeration
\item It has items
\item The items are numbered
\end{enumerate}
\[
	f(x)=\sum_{i=0}^\infty \frac{f^{(i)}(x_0)}{i!}(x-x_0)^i
\]
\end{frame}

\begin{frame}{Theorems and environments}
\begin{theorem}[Sample theorem]
This presentation is essentially useless.
\end{theorem}
\begin{proof}
This proof is essentially incorrect.
\end{proof}
\end{frame}

\begin{frame}{Example slide with Title}
\begin{example}
Major problem.
\end{example}
\begin{solution}
Minor nuisance.
\end{solution}
\end{frame}

\begin{frame}[plain]{Plain frame with title}
\lipsum[1]
\end{frame}
\end{document}