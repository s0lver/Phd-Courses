\section{Problem statement}

\subsection{Preamble}

\begin{frame}{Problem description}
  \begin{block}{Current state of energy management}
    \begin{itemize}
      \item MSA access sensors in a continuous way over long periods of time.
      \item Sensors usage impacts directly on battery.
      \item Current smart devices' processors are designed to manage the heavy interaction with the user and the execution of mobile apps.
      \item A continuous sensor reading is out of their current objectives \citep{Priyantha2011}.
      % \item Current mobile platforms do not include mechanisms to perform periodical readings from sensors.
      \item API’s\footnote{API refers to Application Programming Interface.} by manufacturers only accomplish generic tasks like turning on – off sensors
    \end{itemize}
  \end{block}
\end{frame}

\begin{frame}{Problem description}
  \begin{block}{What do we need?}
    \begin{itemize}
      % \item API’s\footnote{API refers to Application Programming Interface.} by manufacturers only accomplish generic tasks like turning on – off sensors
      \item \textbf{High level information about user's context remains ignored}.
      \item A special framework to generate smart policies for continuous sensor access.
      \item This framework should consider:
        \begin{itemize}
          \item Mobile app requirements (e. g. the precision in the sensor data collection).
          \item Mobile device constraints (e. g. the current level of battery).
          \item Threshold values for performing a smart sensor usage (e. g. the lowest battery level for avoiding a permanent sensor usage).
        \end{itemize}
    \end{itemize}
  \end{block}
\end{frame}

\begin{frame}{Problem description}
  \begin{block}{A possible solution is}
    \begin{itemize}
      \item A policy is a high level concept that defines the usage sensors should observe to keep low energy consumption and fulfill mobile app requirements.

      \item The \emph{smartness} of policies is achieved by leveraging information about the user’s context obtained from sensors data.
      \begin{itemize}
        \item The user’s context can be recognized by employing a pattern identifier mechanism that is fed by raw data collected by sensors.

        \item The pattern becomes the descriptor of user’s context, and is the input for a policy generator mechanism that produces the policy to adapt the sensor usage, reduce the energy consumption and achieve mobile app objectives.
      \end{itemize}
      
    \end{itemize}
  \end{block}
\end{frame}


\subsection{Pattern identification}

\begin{frame}{Problem statement}
  \begin{exampleblock}{Pattern identification}
    Given a set $V = \left\{v_{1}, v_{2}, \dotsc, v_{n}\right\}$ of data values read from sensor $S$ in the time interval $T = \left\{t_{1}, t_{2}, \dotsc, t_{n}\right\}$, find the behavior pattern $Pattern_{S}$ that represents the activity of user.

    \begin{equation}
      PatternIdentifier( V ) \longrightarrow{} Pattern_{S} \in Patterns
    \end{equation}

    Where $Patterns$ is a set of patterns that represent an interesting state in the user activity.
  \end{exampleblock}
\end{frame}


\subsection{Policy generation}

\begin{frame}{Problem statement}
  \begin{exampleblock}{Policy generation}
    Given the pattern $Pattern_{S}$ detected in data from sensor $S$, parameters for assigning weight to energy $eh$ and precision $ph$, and physical constraints status $pc$ of a mobile device, find a policy to adapt the duty cycle of sensors.

    \begin{equation}
      PolicyGeneration( Pattern_{S}, eh, ph, pc ) \longrightarrow{} DutyCycle_{S}
    \end{equation}
  \end{exampleblock}
\end{frame}