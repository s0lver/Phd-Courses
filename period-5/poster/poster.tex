%!TEX program = lualatex
\documentclass[25pt]{tikzposter}
\usetheme{Envelope}
\usebackgroundstyle{Rays}
%\usepackage{emoji}
%\usepackage[unicode]{hyperref}
\usepackage[math]{kurier}
\geometry{paperwidth=100cm, paperheight=130cm}
\makeatletter
\setlength{\TP@visibletextwidth}{\textwidth-2\TP@innermargin}
\setlength{\TP@visibletextheight}{\textheight-2\TP@innermargin}
\makeatother

\title{\parbox{\linewidth}{\centering Smart Usage of Context Information for the Analysis, Design, and Generation of Power-Aware Policies for Mobile Sensing Apps}}
\author{Rafael Pérez Torres, Dr. César Torres Huitzil, Hiram Galeana Zapién Phd}
\institute{LTI Cinvestav Tamaulipas}
\titlegraphic{\includegraphics[scale=0.8]{./images/cinvestav-logo-no-text-white}}

\makeatletter
\renewcommand\TP@maketitle{%
   \centering
   \begin{minipage}[b]{0.7\linewidth}
   % \begin{minipage}[b]{0.8\linewidth}
        \centering
        \color{titlefgcolor}
        {\bfseries \Huge \sc \@title \par}
        \vspace*{1em}
        {\huge \@author \par}
        \vspace*{1em}
        {\LARGE \@institute}
    \end{minipage}%
    \tikz[remember picture,overlay]\node[scale=0.8,anchor=east,xshift=-45cm,yshift=6cm,inner sep=0pt] {%
    % \tikz[remember picture,overlay]\node[scale=0.8,anchor=east,xshift=-0.56\linewidth,yshift=3.9cm,inner sep=0pt] {%
       \@titlegraphic
    };
}
\makeatother

\begin{document}
\maketitle
\block{Resumen}{
\begin{itemize}
	\item La popularidad de los smartphones se debe a sus capacidades en constante incremento de cómputo, comunicaciones y posibilidad de monitorear el entorno a través de sensores.
	\item En particular, el uso de los sensores embebidos y la información extraíble a partir de sus datos permite mejorar la interacción con el usuario y ser conscientes del contexto.
	\item Por ejemplo, los servicios basados en ubicación permiten adaptar el comportamiento del dispositivo de acuerdo a su ubicación, física y semántica.
	\item No obstante, el uso de los sensores ocasiona un elevado consumo de energía, recurso escaso en este tipo de plataformas.
	\item Nuestra propuesta es utilizar la información del contexto obtenida de los sensores para alimentar políticas que realicen un uso adaptativo de los proveedores de ubicación del smartphone.
\end{itemize}
}
% \note[targetoffsetx=-.05\textwidth,targetoffsety=0.1cm,innersep=.4cm,angle=-45,connection]{Tobi}

\block{Problema}{
	\begin{enumerate}
		\item Utilizar datos de los sensores para identificar y \textbf{aprender} la actividad del usuario así como su ubicación.
		\item Utilizar la información aprendida para mejorar el uso de los proveedores de ubicación (GPS) del dispositivo, considerando el compromiso entre \emph{precisión - uso de energía}.
	\end{enumerate}
}

\block{Solución}{
Figura general aquí
}

\begin{columns}
\column{.333}
\block{Event-Driven Systems}{

}

\column{.333}
\block{Physics' perspective of motion}{
	
}

\column{.333}
\block{Cognitive Dynamic Systems}{
	
}

\end{columns}

\block{Resultados preliminares}{
	Tobi
}

\end{document}